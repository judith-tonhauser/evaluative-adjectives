\documentclass[11pt,fleqn]{article}
\usepackage[margin=1in,top=1in,bottom=1in]{geometry}
\usepackage{mathtools}
\usepackage{longtable}
\usepackage{enumitem}
\usepackage{hyperref}
\usepackage[dvips]{graphics}
\usepackage[table]{xcolor}
\usepackage{amssymb}
\usepackage{float}
\usepackage{booktabs}

\usepackage[normalem]{ulem}

\usepackage{multicol}
\usepackage{txfonts}
\usepackage{amsfonts}
\usepackage{natbib}
\usepackage{gb4e}
\usepackage[all]{xy}
\usepackage{rotating}
\usepackage{tipa}
\usepackage{multirow}
\usepackage{authblk}
\usepackage{url}

\def\bad{{\leavevmode\llap{*}}}
\def\marginal{{\leavevmode\llap{?}}}
\def\verymarginal{{\leavevmode\llap{??}}}
\def\swmarginal{{\leavevmode\llap{4}}}
\def\infelic{{\leavevmode\llap{\#}}}

\setlength{\parindent}{.3in}
\setlength{\parskip}{0ex}

%\renewcommand{\baselinestretch}{1.2}

\newcommand{\yi}{\'{\symbol{16}}}
\newcommand{\nasi}{\~{\symbol{16}}}
\newcommand{\hina}{h\nasi na}
\newcommand{\ina}{\nasi na}

\newcommand{\foc}{$ {\mbox{\small F}}$}

\hyphenation{par-ti-ci-pa-tion}

\setlength{\bibhang}{0.5in}
\setlength{\bibsep}{0mm}
\bibpunct[:]{(}{)}{,}{a}{}{,}

\newcommand{\6}{\mbox{$[\hspace*{-.6mm}[$}} 
\newcommand{\9}{\mbox{$]\hspace*{-.6mm}]$}}
\newcommand{\sem}[2]{\6#1\9$^{#2}$}
\renewcommand{\ni}{\~{\i}}

\newcommand{\citepos}[1]{\citeauthor{#1}'s \citeyear{#1}}
\newcommand{\citeposs}[1]{\citeauthor{#1}'s}
\newcommand{\citetpos}[1]{\citeauthor{#1}'s (\citeyear{#1})}



\title{Evaluative adjective sentences: \\ A question-based analysis of projection}

%\thanks{For helpful comments and critical feedback, we thank David Beaver, Ashwini Deo, Lauri Karttunen, Stanley Peters, Craige Roberts,
%Stephanie Solt and the audiences at the following venues: the 6th meeting of the DFG Network
%{\em Questions in Discourse} at the University of G\"ottingen, 
%the University of Stuttgart, the 16th Sklarska Poreba workshop, the
%Redrawing Pragmasemantic Borders workshop at the University of
%Groningen, the University of Cologne, Oxford University, the University of T\"ubingen,  the University of G\"ottingen, the {\em
%Experimental Pragmatics} 2015 conference in Chicago, The Ohio State
%University and Cornell University. We thank Willy Cheung for his assistance with the corpus study and Timo Roettger for consultation on the Bayesian regression analysis. We gratefully acknowledge
%financial support from the {\em National Science Foundation} (grants
%BCS-0952571 and BCS-1452674), the Alexander von Humboldt Foundation and the Ohio State
%University College of Arts and Sciences Targeted Investment in
%Excellence initiative.}}

\author{}
%\author[$\circ$]{Judith Tonhauser}
%\author[$\bullet$]{Judith Degen}
%\author[$\circ$]{Marie-Catherine de Marneffe}
%\author[$\star$]{Mandy Simons}
%
%\affil[$\circ$]{The Ohio State University}
%\affil[$\bullet$]{Stanford University}
%\affil[$\star$]{Carnegie Mellon University}
%
%\renewcommand\Authands{ and }



\newcommand{\jt}[1]{\textbf{\color{blue}JT: #1}}
\newcommand{\jd}[1]{\textbf{\color{red}[jd: #1]}}
\newcommand{\mcdm}[1]{\textbf{\color{teal}[mcdm: #1]}}


\begin{document}

\maketitle

\vspace*{-.7cm}

% length at beginning of revisions: 27,235 (monterrey), main body: 20,803
% length at end of revisions

\begin{abstract}

Two contents of evaluative adjective sentences, like {\em Kim was smart to watch the movie}, are the prejacent (that Kim watched the movie) and the generalization (that the degree to which Kim watching the movie is smart was higher than the contextual standard of {\em smart}). The prejacent is standardly analyzed as a presupposition
(e.g., \citealt{norrick78,barker02,oshima09b,kertz2010}). This paper argues against such analyses of the prejacent because, among other things, they do not capture an interaction between the prejacent and the generalization that has not yet been observed for projective content: when the prejacent projects, the generalization does not, and when the prejacent does not project, the generalization does. We develop an analysis according to which the prejacent is not a lexically specified presupposition but projects if it is not at-issue with respect to the question addressed by the utterance of the evaluative adjective sentence. In addition to capturing the interaction between the prejacent and the generalization, our question-based projection analysis significantly extends previous such analyses (e.g., \citealt{beaver-clark08,best-question,brst-ar}) by incorporating a novel constraint on the question addressed by an utterance: the more the truth of an utterance content is taken to follow from the common ground, the less likely the question is about that content. We provide experimental evidence for the proposed analysis and argue that our analysis improves on that of \citealt{karttunen-etal2014}, according to which evaluative adjectives are systematically ambiguous. 

\end{abstract}


\section{Introduction}\label{s1}

In an evaluative adjective sentence (EAS) like (\ref{f}), an
evaluative adjective\footnote{Evaluative adjectives are adjectives like {\em stupid, rude} or {\em fortunate} that, in sentences like {\em Sam was stupid/rude/fortunate}, convey the speaker's positive or negative evaluation of or attitude towards the denotation of the subject noun phrase.} ({\em stupid}) subcategorizes for a non-pleonastic subject noun
phrase ({\em Feynman}) and a {\em to-}infinitive ({\em to
dance on the table}); the subject of the predicate ({\em be stupid}) is the understood subject of the {\em to-}infinitive (e.g., \citealt{wilkinson70, norrick78,barker02,kertz2010}).  One of the contents standardly discussed in the literature is (what we refer to as) the prejacent: in (\ref{f}), the prejacent is that Feynman danced on the table.

\begin{exe} 
\ex\label{f} Feynman was stupid to dance on the table. \hfill (\citealt[18]{barker02})
\end{exe} 
The prejacent has traditionally been analyzed as a presupposition (e.g., \citealt{norrick78,barker02,oshima09b,kertz2010}). Thus, under formal analyses in this tradition, evaluative adjectives like {\em stupid} lexically specify that the prejacent
must be entailed by or satisfied in the common ground of the interlocutors in
order for an EAS to be interpretable (e.g.,
\citealt{heim83,vds92}). Such analyses are motivated by and straightforwardly account for the variants of (\ref{f}) in (\ref{f2}) in which the prejacent may project over entailment-canceling operators, such as negation in (\ref{f2}a), a polar question in (\ref{f2}b), the possibility adverb {\em perhaps} in (\ref{f2}c) and the antecedent of the conditional in (\ref{f2}d). That is, speakers who utter (\ref{f2}a-d) may be taken to be committed to the truth of the prejacent, that Feynman danced on the table, even though the evaluative adjective is embedded under an entailment-canceling operator. 

\begin{exe}

\ex\label{f2} 

\begin{xlist}

\ex Feynman wasn't stupid to dance on the table.

\ex Was Feynman stupid to dance on the table?

\ex Perhaps Feynman was stupid to dance on the table.

\ex If Feynman was stupid to dance on the table, then tell him. \hfill (\citealt[18f.]{barker02})

\end{xlist}
\end{exe}

Recently, \citet{karttunen-etal2014} provided naturally occurring examples that show that utterances of sentences in which the evaluative adjective is embedded under negation can receive an interpretation according to which the prejacent does not project, i.e., can be interpreted in the scope of negation. For instance, the speaker of (\ref{nat}a) is not committed to living close to their parents and instead communicates that they do not live close to their parents. Likewise, the speaker of (\ref{nat}b) communicates that they did not go stumbling through the junkyard and get hurt.\footnote{Non-projection of the prejacent is also observed with  other entailment-canceling operators, such as the antecedent of a conditional in B's utterance in (ia) or the possibility modal {\em perhaps} in (ib). While this paper limits its attention to EASs embedded under negation, we expect the analysis to generalize  to other entailment-canceling operators.

\begin{exe}
\exi{(i)}
\begin{xlist}
\ex 
\begin{xlist}
\exi{A:} I wonder how many people would be banned for using SAM on Killing Floor.
\exi{B:} If they were stupid to leave it running then maybe a few.\\ \url{steamcommunity.com/app/730/discussions/0/540744934462316309/}
\end{xlist}

\ex I am seaching [sic] for a remote cabine in Finland that is available for renting. [...]
Region wise I would prefer Lapland or Lakeside. Perhaps we are lucky to see northern lights. \\ \url{www.tripadvisor.com/ShowTopic-g189896-i442-k11448709-Remote rental cottage-Finland.html}

\end{xlist}
\end{exe}

}

\begin{exe}
\ex\label{nat} \citealt[235]{karttunen-etal2014}
\begin{xlist}
\ex I wasn't fortunate to live extremely close to my Mom and Dad for most of my adult life. The closest was when I was in Denver and they were in Garden City, KS.

\ex  Now I knew someone was in the junkyard and the cold wind was
carrying the cries. I wasn't stupid to go stumbling through the
junkyard in the dark and get hurt.

\end{xlist} \end{exe} 

Before discussing why such examples are problematic for analyses of the prejacent as a lexically specified presupposition, we would first like to acknowledge that there is variation in the population of native speakers of American English: whereas speakers are generally able to interpret EASs in which the prejacent does not project, i.e., are able to retrieve the intended interpretations of examples like (\ref{nat}), a sizeable portion strongly prefers to realize such interpretations with variants that include {\em enough}, as in {\em I wasn't stupid enough to go stumbling through the junkyard in the dark} for (\ref{nat}b); see also \citealt{karttunen2013} and \citealt{karttunen-etal2014} for this observation. Nevertheless, EASs in which the prejacent does not project are part of American English, as evidenced by the existence of naturally occurring examples like (\ref{nat}). Additional evidence comes from the fact that there are native speakers who judge such EASs to be perfectly acceptable\footnote{The audiences to which this research was presented over the years always included native speakers that judged examples like (\ref{nat}) to be acceptable. More systematic evidence for the existence of such speakers comes from the acceptability rating study presented in Appendix \ref{s-acc}: of the 94 self-reported native speakers of American English that participated in the study, about 20-30\% judged negated evaluative adjective sentences without {\em enough} to be acceptable under a non-projecting interpretation of the prejacent.} and who produce examples like (\ref{nat}). We therefore assume that native speakers of American English are generally able to retrieve both interpretations of EASs even if they might not produce EASs in which the prejacent does not project. Our goal in this paper is to analyze the interpretation of EASs; we briefly return to the observed production variation after developing our analysis.

To account for examples like (\ref{nat}), in which the prejacent does not project, presupposition analyses appeal to local accommodation, a process whereby a presupposition is added to a local context, such as that created by negation: presuppositions can be locally accommodated if the default global accommodation -- adding the presupposition to the common ground of the interlocutors -- would result in a contradiction, uninformativity or problems with
binding (\citealt{heim82,vds92}). In example (\ref{nat}a), for instance, the prejacent, that the speaker lived close to their parents, is locally accommodated under negation because global accommodation would result in a contradiction: 
according to the context, the closest that the speaker lived to their parents was when the speaker lived in Denver, Colorado, i.e., about 300 miles (480 km) away from their parents in Garden City, Kansas. The prejacent is correctly predicted to be locally accommodated in (\ref{nat}a) and thereby is not a commitment of the speaker.

Negated evaluative adjective sentences (NEASs) like (\ref{nat}) point to two problems for analyses according to which the prejacent is a presupposition. The first problem concerns the prejacent. In example (\ref{nat}b), globally accommodating the prejacent does not result in a contradiction, uninformativity or problems with binding: if the prejacent, that the speaker went stumbling in the junkyard, was added to the common ground, (\ref{nat}b) would mean that the speaker knew that someone was in the junkyard, that they stumbled through the junkyard and got hurt, and that the speaker does not consider these actions stupid (perhaps because these actions led to the person in the junkyard receiving help). Because such an interpretation does not result in a contradiction, uninformativity or problems with binding, analyses of the prejacent as a presupposition incorrectly  predict that the prejacent of (\ref{nat}b) is globally accommodated, i.e., is a commitment of the speaker.\footnote{It may be possible to account for the non-projection of the prejacent on the basis of plausibility considerations: \citealt[162]{vonfintel08}, for instance,  proposed that ``what gets accommodated depends on the best guess of the listeners about what the speaker might have intended as the adjustment to the common ground that would admit the asserted sentence''. The cues to the projection of the prejacent identified in this paper may help flesh out the reasoning process that listeners/readers undergo in interpreting EASs. Plausibility considerations do not, however, address the second problem that presuppositional analyses face.}

The second problem concerns content other than the prejacent: here, the two prior analyses that explicitly consider content other than the prejacent, \citealt{oshima09b} and \citealt{barker02}, do not make sufficiently strong predictions. Consider first \citealt{oshima09b}. On this analysis, the prejacent is a presupposition and what is asserted is the following complex content: from the prejacent it can be inferred that the denotation of the subject is in the extension of the evaluative adjective (p.371). That is, (\ref{f2}a) is predicted to presuppose that Feynman danced on the table and to assert that it cannot be inferred from Feynman dancing on the table that Feynman is stupid. Similarly then, (\ref{nat}b) is predicted to convey that the speaker did not stumble through the junkyard (if the prejacent is locally accommodated) and to assert that it cannot be inferred from the speaker stumbling through the junkyard that the speaker is stupid. This assertion is too weak: (\ref{nat}b) means that the speaker stumbling through the junkyard would be stupid. 

Next consider \citealt{barker02}. This analysis considers the prejacent, which is taken to be presupposed, and a content that we refer to as the generalization: the generalization of (\ref{f2}a) is that the degree to which Feynman dancing on the table is stupid was higher than the contextual standard of {\em stupid}; that of (\ref{nat}b) is that the degree to which the speaker stumbling through the junkyard in the dark is stupid was higher than the contextual standard of {\em stupid}. On \citetpos{barker02} dynamic semantic analysis, the update effect of a NEAS like (\ref{f2}a) is to check that, for each world, the presupposed prejacent is true and to filter out those worlds in which the generalization is false; thus, only those worlds remain in which Feynman dancing on the table was not considered stupid. Similarly then, the update effect of (\ref{nat}b), in which the prejacent is locally accommodated, is to filter out those worlds in which the contextual standard of {\em stupid} is too high for the speaker's participation in the event of stumbling through the junkyard to have counted as stupid. The worlds that remain are ones in which the speaker did not stumble through the junkyard and in which the speaker stumbling through the junkyard was not stupid. This, again, is not what (\ref{nat}b) is understood to mean. 

The examples in (\ref{f2}) and (\ref{nat}) reveal a remarkable interaction that has not yet been observed for projective content and that was not noted in \citealt{karttunen-etal2014}: when the prejacent projects, the generalization does not, as in (\ref{f2}b), and when the prejacent does not project, the generalization does, as in (\ref{nat}b). To illustrate that this behavior is strikingly different from the interpretation of utterances of other sentences that give rise to projective content consider  (\ref{stop}). When the content of the complement of {\em know}, that the meeting was canceled, projects, what is denied is Sam's knowledge of this content. The critical difference between EASs and examples like (\ref{stop}) comes out when the content of the complement does not project, as is brought out, for instance, by continuing (\ref{stop}) with {\em \ldots he, like all of us, is in the dark about whether the meeting will take place}. In this case, Sam's knowledge of the content that the meeting was canceled is still denied and the speaker is not taken to be committed to either its truth or its falsity. Thus, the two contents of (\ref{stop}) do not exhibit the interaction of the projectivity of the prejacent and the generalization.

\begin{exe}
\ex\label{stop} Sam doesn't know that the meeting was canceled.
\end{exe}
This observation has strong implications for analyses of EASs according to which the prejacent is a lexically specified presupposition. First, when the prejacent is locally accommodated under negation, the speaker is committed to the falsity of the prejacent; this is in contrast to a locally accommodated factive presupposition, for which the speaker is not committed to its truth or falsity. Second, when the prejacent is locally accommodated under some operator, the generalization is not interpreted under that operator, in contrast to the attitude ascription with {\em know}, which is always interpreted in the scope of the operator. These observations are problematic for advocates of analyses according to which the prejacent is a lexically specified presupposition and projection is assumed to be governed by the standard mechanisms of presupposition projection (e.g., \citealt{heim83,vds92}). 

\citepos{karttunen-etal2014} analysis of EASs made progress over previous analyses; it did so by assuming that evaluative adjectives are ambiguous between the two schematic lexical entries in (\ref{lex}): the prejacent is specified as presupposed in the lexical entry {\em adj}$_1$ in  (\ref{lex}a) but not in that of {\em adj}$_2$ in (\ref{lex}b). 

\begin{exe}
\ex\label{lex} \citealt[249]{karttunen-etal2014}: Presuppositions and assertions of EASs of the form `NP was Adj to VP'
\begin{xlist}
\ex {\em adj}$_1$
\\ Presupposed content: NP VPed
\\ Asserted content: For NP to VP would be Adj

\ex {\em adj}$_2$
\\ Presupposed content: For NP to VP would be Adj \& for NP not to VP would not be Adj
\\ Asserted content: What NP did about VPing was Adj

\end{xlist}
\end{exe}
These two lexical entries correctly predict an interpretation of the NEAS in (\ref{f2}a) in which the prejacent projects and an interpretation of the NEAS in (\ref{nat}b) according to which the prejacent does not project. Furthermore, the lexical entry in (\ref{lex}b) predicts that the NEAS in (\ref{nat}b) has an interpretation according to which it is presupposed that for the speaker to stumble through the junkyard would be stupid and for the speaker to not stumble through the junkyard would not be stupid; thus, in contrast to \citepos{oshima09b} and \citepos{barker02} analyses, this analysis correctly predicts that (\ref{nat}b) conveys that the speaker stumbling through the junkyard would be stupid. Finally, Karttunen et al.'s analysis captures the interaction between the projection of the prejacent and the generalization: when the prejacent projects, as in the lexical entry in (\ref{lex}a), the generalization does not, by virtue of being coded in the asserted content;  when the prejacent does not project, as in the lexical entry in (\ref{lex}b), the generalization does, by virtue of being part of the presupposed content.

There are, however, concerns with Karttunen et al.'s analysis. First, because it is not formalized, it is not clear that it makes correct predictions about content other than the prejacent. According to the lexical entry in (\ref{lex}a), what is asserted is that it is not the case that for Feynman to dance on the table would be stupid: \citet[248]{karttunen-etal2014} assume that it follows from this asserted content and the presupposed prejacent that it is not the case that Feynman dancing on the table was stupid. According to the lexical entry in (\ref{lex}b), what is asserted is that it is not the case that what the speaker did about stumbling through the junkyard was stupid. That the prejacent is negated (the speaker didn't stumble through the junkyard) follows, according to \citealt[249]{karttunen-etal2014}, from what is presupposed and what is asserted. A second concern is that Karttunen et al.'s analysis comes at a high cost: to derive the two interpretations, evaluative adjectives are systematically ambiguous. Finally, in addition to hardwiring the projection of the prejacent into one lexical entry but not the other, the analysis also hardwires the interaction between the projection of the prejacent and the generalization. Preferably, this interaction would fall out of the analysis. 

In this paper, we develop an analysis that derives the two interpretations of EASs and the interaction between the projection of the prejacent and the generalization from a single lexical entry for evaluative adjectives. This is achieved, in a nutshell, by making the projection of the prejacent and the generalization sensitive to their discourse status and not lexically specifying them as presupposed. In doing so, our analysis builds on previous analyses according to which projection of content is derived from its discourse status, such as being backgrounded or not-at-issue (e.g., \citealt{abrusan2011,abrusan2013,abrusan2016,brst-salt10,best-question,brst-ar,tbd-variability,tonhauser-etal-sub23}). Analyses in this tradition have been developed for utterance content whose projectivity is sensitive to the discourse context, such as the prejacent of manner adverbs (e.g., that Sam ran in {\em Sam ran quickly}), the pre-state content of {\em stop} (e.g., that Sam whistled in {\em Sam stopped whistling}) or the content of the complement of factive predicates like {\em know} in (\ref{stop}). Such analyses predict not only that the projectivity of such content is sensitive to its discourse status, but also correctly predict that it is less projective than content whose projection is conventionally specified, like the content of non-restrictive relative clauses or the projective content of anaphoric expressions like pronouns or {\em too} (for discussion see, e.g., \citealt{kadmon01,potts05,abrusan2011,abrusan2016,brst-lang11,tbd-variability,tonhauser-etal-sub23}). What it means for one content to be less projective than another is that listeners are less likely to take the speaker to be committed to the truth of the first content than the second. To illustrate, consider the examples in (\ref{proj-var}), in which the content of the complement of {\em know} in (\ref{proj-var}a) and the content of the non-restrictive relative clause in (\ref{proj-var}b) is that the meeting was canceled. Both contents are projective: a speaker who utters the first sentences of these examples may be taken to be committed to the meeting having been canceled. However, listeners are less likely to take the speaker of (\ref{proj-var}a) to be committed to the meeting having been cancelled than the speaker of (\ref{proj-var}b): evidence for this difference in projection comes from the fact that the second sentence is an acceptable continuation in (\ref{proj-var}a) but not (\ref{proj-var}b). For experimental evidence that projection is not a binary, categorical property of utterance content but rather a gradient one see \citealt{smith-hall11,xue-onea11} and \citealt{tbd-variability,tonhauser-etal-sub23}.\footnote{According to \citealt[498f.]{tbd-variability}, there are at least two interpretations of what it means for projectivity to be a gradient rather than a categorical, binary property of utterance content: 

\begin{quote}

``On a first interpretation, a listener's (or reader's) judgment that a content is projective to a certain extent means that the listener takes the speaker (or writer) to be committed to the content to that extent. On this interpretation, projectivity being a gradient property is a consequence of speaker commitment being a gradient property. On a second interpretation, a listener's judgment that a content is projective to a certain extent reflects the probability with which they believe the speaker to be committed to the content. On this interpretation, speaker commitment may be a binary, categorical property and projection variability arises from the listener's uncertainty about the whether the speaker is committed."

\end{quote}

Like \citealt{tbd-variability}, we remain agnostic about the underlying interpretation of projectivity as a gradient property, though our discussion of projection variability will be in line with the first interpretation.}


\begin{exe}

\ex\label{proj-var} 

\begin{xlist}

\ex Sam doesn't know that the meeting was canceled. He, like all of us, is in the dark about whether the meeting will take place.

\ex It's not the case that the meeting, which was canceled, was going to take place in the conference room. \#I am in the dark about whether the meeting will take place.

\end{xlist}
\end{exe}

Empirical motivation for adopting an analysis of the prejacent of EASs according to which its projectivity is not derived from its status as a lexically specified presupposition but from its discourse status comes from the observation that the prejacent of EASs is not highly projective.  \citepos{tbd-variability} experimental investigation found that the projectivity of the prejacent of EASs with {\em stupid} was significantly lower than that of non-restrictive relative clauses and appositive content, as well as of the content of the complement of the factive predicates {\em be annoyed} and {\em know}. In naturally occurring data, too, the prejacent is not highly projective: in a corpus-based web study, we collected `certainty' ratings on a 7-point Likert scale from 226 native speakers of American English for the prejacents of 59 naturally occurring NEASs; the higher the certainty rating, the more projective the prejacent. Figure \ref{f-corpus}, which plots the mean certainty ratings of the 59 NEASs by evaluative adjective, shows that the prejacent exhibits projection variability: for some NEASs, the prejacent is highly projective, for some it is weakly projective, and for others it is in-between. Across the 59 NEASs, the prejacent is not highly projective:  the mean certainty rating was only 3.2.\footnote{In contrast to the prejacent of EASs like {\em Kim was smart to watch the movie}, the prejacent of sentences like {\em It was smart of Kim to watch the movie} appears to be more highly projective according to native speaker intuitions, and, therefore, may be best analyzed as a lexically specified presupposition, as in \citealt{oshima09b}.} Details on this corpus-based web study are given in Appendix \ref{a-corpus}.

\begin{figure}[h!]
\centering

\includegraphics[width=.6\paperwidth]{../exp1-corpus-study/graphs/mean-response-by-item-and-adj}

\caption{Mean certainty ratings of 59 naturally occurring NEASs by evaluative adjective. Error bars indicate bootstrapped 95\% confidence intervals.}\label{f-corpus}

\end{figure}

In the next section, we build on \citepos{best-question} question-based analysis of projective content to develop an analysis that derives the projectivity of the prejacent and the generalization from their status as not-at-issue with respect to the question addressed by the EAS. Section \ref{s3} provides experimental evidence for two predictions of the analysis. After briefly considering the aforementioned interspeaker variation in section \ref{s5}, the paper concludes in section \ref{s6}.

\section{A question-based analysis of the projective content of EASs}\label{s2}

In this section, we develop a question-based projection analysis for EASs. We start in section \ref{s21} by introducing the question-based analysis developed in \citealt{best-question} for the content of the complement of factive predicates. In section \ref{s22}, we extend this analysis to EASs and show how the  interaction between the projectivity of the prejacent and the generalization falls out of the analysis. In section \ref{s23}, we extend the analysis by incorporating a constraint on the questions addressed by utterances of EASs.

\subsection{Utterance content projects if it is not at-issue with respect to the Discourse Question}\label{s21}

The content of the complement of factive predicates like {\em know} or {\em discover} has long been observed to be projective (e.g., \citealt{kiparsky-kiparsky70,karttunen71b}). Thus, the content of the complement of {\em discover} can be understood as a commitment of the speaker/writer, as  in (\ref{discover}a), but it does not have to be, as in (\ref{discover}b). See also the discussion around example (\ref{stop}) with {\em know} in section \ref{s1}.

\begin{exe}
\ex\label{discover}
\begin{xlist}
\ex Caroline stepped forward after a moment. ``There is, of course, also the issue of Alex and the other Enhanced to tackle. If anyone {\bf discovers} that Alex is here, it will be a disaster difficult to avert.'' Alex frowned. \hfill L.C.\ Mawson, {\em Pandora: Freya Snow}, \#13

\ex {[}Mattress springs] also work well to deter rabbits \& foxes from digging into the chook-pen (Hen-run). Dig a shallow trench the width of a single mattress, then place the springs flat into the trench. Drive your fence posts in the mid-line, so half the spring is outside \& half inside the pen. I haven't tried this with wombats, though \& if anyone {\bf discovers} that the method is also wombat-proof, I'd really like to know. \hfill (\citealt[79]{beaver-belly})

\end{xlist}
\end{exe}

Traditionally, projection of the content of the complement has been derived by specifying the content as presupposed in the lexical entry of the factive predicate (e.g., \citealt{heim83,vds92}); as discussed in section \ref{s1}, non-projection under such analyses is attributed to local accommodation. In view of the observation that the projectivity of the content of the complement is weaker than that of other projective content, some analyses have abandoned the assumption that the content projects because it is lexically specified as presupposed: on some analyses, projection is derived from a lexically specified set of alternatives to the factive predicate in combination with pragmatic principles (e.g., \citealt{abusch02,abusch10,romoli2015}) and on others it is derived from the discourse status of the content of the complement (e.g., \citealt{abrusan2011,abrusan2016,brst-salt10,best-question}). Analyses that derive the projection of the content of the complement from its discourse status are empirically motivated by the observation that projection is sensitive to information structure. To illustrate, consider the examples in (\ref{beaver}): \citet{beaver-belly} hypothesized that the content of the complement of {\em discover} is less projective in (\ref{beaver}a), which is realized with focus on {\em plagiarized}, than in (\ref{beaver}b), which is realized with focus on {\em discover}. For experimental evidence for this hypothesis see \citealt{cummins-rohde2015,tonhauser-salt26,djaerv-bacovcin-salt27} and \citealt{mahler-nels}.

\begin{exe}
\ex\label{beaver}
\begin{xlist}
\ex If the T.A.\ discovers that your work is [plagiarized]$_{\mbox{F}}$, I will be forced to notify the Dean.

\ex If the T.A.\  [discover]$_{\mbox{F}}$ that your work is plagiarized, I will be forced to notify the Dean. \\ \hspace*{.2cm} \hfill (adapted from \citealt[93]{beaver-belly})

\end{xlist}
\end{exe}

To capture the information-structure sensitivity of the projection of the content of the complement,  \citealt{brst-salt10,best-question} assumed that the content of the complement is a lexical entailment of sentences with factive predicates that projects if it is not at-issue with respect to the Discourse Question addressed by the utterance of the sentence with the factive predicate.\footnote{Following \citealt{abrusan2011}, lexical entailments are entailments of sentences, not predicates. They differ from logical entailments, which can be derived independently from the meaning of the sentence. For instance, the disjunction of the content of the complement and the proposition that it is raining is a logical entailment of a sentence with a factive predicate, but not a lexical one. For the assumption that at least some projective contents, including presuppositions, are entailments that may project see, e.g., \citealt{gazdar79b,barker02,schlenker10,abrusan2011,abrusan2016} and \citealt{anand-hacquard2014}.} The Discourse Question ``provides the topic of a segment of discourse and imposes relevance constraints on conversational contributions'' (\citealt[192]{best-question}). Utterance content is at-issue with respect to the Discourse Question of the utterance if the content addresses the Discourse Question, i.e., entails at least a partial answer to the question (\citealt{roberts12}/1996). To illustrate, consider the content of the complement of {\em discover} in B's utterances in (\ref{discover2}a) and (\ref{discover2}b), that Harriet was at Princeton for a job interview. In the these examples, the Discourse Questions that B's utterances address are made explicit by A's interrogative utterances. In (\ref{discover2}a), the content of the  complement of {\em discover} is at-issue because it addresses the Discourse Question: that Harriet was at Princeton for a job interview is an answer A's interrogative utterance of where Harriet was yesterday. In (\ref{discover2}b), on the other hand, the content of the complement is not at-issue because it is not an answer to A's interrogative utterance; rather, here the Discourse Question is addressed by the main clause content of B's utterance.


\begin{exe}
\ex\label{discover2}
\begin{xlist}
\ex
\begin{xlist}
\exi{A:} Where was Harriet yesterday?
\exi{B:} Henry discovered that she was at Princeton for a job interview.
\end{xlist}

\ex
\begin{xlist}
\exi{A:} Why is Henry in such a bad mood?
\exi{B:} He discovered that Harriet was at Princeton for a job interview.
\\ \hspace*{.2cm} \hfill (examples adapted from \citealt[1035]{simons07})
\end{xlist}

\end{xlist}
\end{exe}

Now consider the example in (\ref{discover22}), in which  {\em discover} is embedded in a polar question: consequently, the speaker is not taken to be committed to the content of the main clause; rather, the speaker is asking whether Henry discovered the content of the complement.

\begin{exe}
\ex\label{discover22} Did Henry discover that Harriet was at Princeton for a job interview?
\end{exe}
In contrast to the main clause content, the content of the complement of {\em discover} may project out of the polar question, as noted above. According to \citealt{best-question}, one of the conditions under which the content of the complement projects is when it is not at-issue with respect to the Discourse Question addressed by the utterance. Consider the examples in (\ref{discover3}), where A's interrogative utterances again make explicit the Discourse Questions addressed by B's utterances. In (\ref{discover3}a), the content of the complement of B's utterance does not address the Discourse Question, i.e., is not at-issue. Here, B is taken to be committed to the content of the complement: to make sense of B's utterance as an answer to A's question, there must be a connection between the possibility of Henry discovering something about Harriet and Henry's mood; an obvious connection is that B takes the content of the complement to be true, i.e., that it is possible that Henry discovered that Harriet was at Princeton for a job interview. Thus, the content of the complement projects. In (\ref{discover3}b), by contrast, the content of the complement addresses the Discourse Question, i.e., is at-issue. Here, B need not be taken to be committed to the content of the complement to make sense of how B's utterance addresses A's question and the content of the complement does not project.

\begin{exe}
\ex\label{discover3}
\begin{xlist}

\ex Context: Henry and Harriet are an academic couple that lives on the West Coast. 
\begin{xlist}
\exi{A:} Why is Henry in such a bad mood?
\exi{B:} Did he discover that Harriet was at Princeton for a job interview?
\end{xlist}

\ex Context: Henry is a nosy colleague of Harriet and well-informed about her whereabouts.
\begin{xlist}
\exi{A:} Where was Harriet yesterday?
\exi{B:} Did Henry discover that she was at Princeton for a job interview?
\end{xlist}

\end{xlist}
\end{exe}
In sum, according to \citealt{best-question}, utterance content projects if it is not at-issue with respect to the Discourse Question addressed by the utterance. The question of why not-at-issue content projects has received several answers. According to \citet{potts05}, it projects because it is contributed to a separate dimension of meaning and, according to \citet{brst-salt10}, because it is not targeted by operators like negation and thereby projects over such operators. In this paper, we follow \citealt{abrusan2011,abrusan2016} in assuming that not-at-issue content is backgrounded and projects as a result of its discourse status.\footnote{We do not adopt Abrus\'an's analysis otherwise because it does not appear to make correct predictions for EASs. \citet{abrusan2011} proposed that lexical entailments that are about the running time of the main event are the default main point, i.e., what we have referred to as the at-issue content. This analysis, however, does not make correct predictions for EASs: given Abrus\'an's notion of aboutness, both the prejacent and the generalization of EASs are about the running time of the main event. For instance, \citet[508]{abrusan2011} took the entailment of (i), that John solved the exercise, to be ``non-accidentally (i.e., necessarily) about the matrix event time''. By the same argument, the prejacent of (ii) would be about the matrix event time, which means that neither the prejacent nor the generalization are predicted by \citealt{abrusan2011} to be the default at-issue content of EASs.

\begin{exe}
\exi{(i)} John managed (at time t$ {\mbox{1}}$) to solve the exercise (at t$ {\mbox{1}}$). \hfill (\citealt[508]{abrusan2011})

\exi{(ii)} John was smart (at time t$ {\mbox{1}}$) to solve the exercise (at t$ {\mbox{1}}$).

\end{exe}} 

In naturally occurring discourse, the Discourse Question addressed by an utterance is more likely to be implicit rather than explicit. When the Discourse Question is implicit, the utterance itself and the discourse context in which it is made provide cues to the Discourse Question, and thereby to the at-issueness and projection of utterance content. For instance, for utterances of sentences with factive predicates, the information structure of the utterance, in particular prosodically marked focus, has been shown to constrain the question addressed by the utterance and, hence, the at-issueness and projection of the content of the complement (e.g., \citealt{beaver01,cummins-rohde2015,tonhauser-salt26,best-question,djaerv-bacovcin-salt27,mahler-nels}). Importantly, when the Discourse Question is implicit, there may be uncertainty on part of the interpreter about the Discourse Question that the speaker intended to address with their utterance and, consequently, about whether a particular utterance content is at-issue. In other words, for any given utterance content, the listener may be more or less likely to assume that it is at-issue. Given that the projectivity of utterance content is also variable, \citealt{tbd-variability} modified \citepos{best-question} proposal, according to which content projects if it is not at-issue, by arguing instead that content projects to the extent that it is not at-issue:

\begin{exe}
\ex\label{gpp} {\bf Gradient Projection Principle:} If content $C$ is expressed by a constituent embedded under an entailment-canceling operator, then $C$ projects to the extent that it is not at-issue. \\ \hspace*{.2cm} \hfill (\citealt[499]{tbd-variability})
\end{exe}

\citealt{tbd-variability} provided experimental evidence for the Gradient Projection Principle on the basis of an experimental investigation 19 projective contents associated with American English expressions. Specifically, \citet{tbd-variability} found a positive correlation between not-at-issueness and projection such that content that is more not-at-issue is also more projective. For instance, as shown in Figure \ref{tbd} for 9 of these 19 contents ($r$ = .85), the prejacent of EASs with {\em stupid} is not only less projective than the content of the complement of {\em know}, but also less not at-issue.

\begin{figure}[!h]

\begin{center}
\includegraphics[width=.55\textwidth]{figures/ai-proj-bytrigger-labels}
\end{center}

\caption{\citealt[509]{tbd-variability}: mean projectivity against mean not-at-issueness by target expression/projective content. Error bars indicate bootstrapped 95\% confidence intervals.}
\label{tbd}
\end{figure}

In the next section, we build on the question-based projection analysis developed in \citealt{best-question} and \citealt{tbd-variability} in developing a question-based projection analysis of EASs.

\subsection{Lexical entailments of evaluative adjective sentences}\label{s22}

We assume, with \citealt{barker02}, that the prejacent and the generalization are lexical entailments of unembedded EASs (but not that the prejacent is lexically specified as presupposed).\footnote{\label{sre}As lexical entailments, the prejacent and the generalization are entailments of evaluative adjective sentences, not of the evaluative adjectives. This is not to say that the evaluative adjectives do not give rise to entailments: as discussed in detail in \citealt[\S4.2]{barker02}, they give rise to subject-related entailments, namely that the denotation of the subject noun phrase is capable of volition and, regarding the situation described by the {\em to-}infinitive, has the power to bring it about and intends for it to come about.  We assume, with \citet{barker02}, that these subject-related entailments are presuppositions that are triggered by the evaluative adjectives.}$^{,}$\footnote{Some EASs with {\em will} appear to not entail the prejacent: for example, some speakers do not judge (i) to entail that Johnson will take what he can get. We thank David Beaver (p.c.) for this point, which we sidestep here.
\begin{exe}
\exi{(i)} With more teams denying interest in Johnson, he will be smart
to take what he can get. \\ ({\em http://www.sportsworldreport.com/articles/28999/20140407/chris-johnson-rumors-ny-jets-release-mike-goodson-free-agent-signs-dallas-cowboys-demarco-murray-rams-falcons.htm})
\end{exe}} 
 Translations of the prejacent and the generalization of an unembedded EAS of the form `NP be{\sc .tense} Adj to VP' are given in (\ref{ent}). In the translation of the prejacent in (\ref{ent}a), the translation of the NP is the constant $np$ (of type $e$, for entities) and of the VP is $VP'$ (of type $\langle e, \langle ev,  t\rangle \rangle$, where $ev$ is the type of eventualities and $t$ is the type of truth values). The run time of the event $e$ of $NP$ $VP$ing, given as $\tau(e)$, is temporally located at the reference time $rt$, whose temporal location is constrained by the tense of the EAS. In the translation of the generalization in (\ref{ent}b), the constant $deg'$ combines with the translation of the adjective and maps it to the contextual standard of the adjective (a degree, type $d$). 

\begin{exe}
\ex\label{ent}  Unembedded EAS of the form `NP be{\sc .tense} Adj to VP'

\begin{xlist}

\ex Prejacent: $\exists e (VP'(np)(e) \wedge at(rt,\tau(e)))$ \\ ``There is an event of $NP$ $VP$ing and the run time of that event is  located at the reference time.''

\ex Generalization: {\em adj}$'(rt,VP'(np),deg'(adj'))$ \\ ``At the reference time, the degree to which events of $NP$ $VP$ing are {\em Adj} is higher than the contextual standard for {\em Adj}.''

%\ex Generalization: $adj'(VP'(np))$ \\ ``The degree to which events of $np$ $vp$ing are $adj$ is higher than the contextual standard for $adj$.''

\end{xlist}

%\ex\label{ent} Two lexical entailments of unembedded evaluative adjective sentences `NP be{\sc .tense} Adj to VP'
%
%\begin{xlist}
%
%\ex Generalization: In VPing NP is/would be Adj
%
%\ex Prejacent: NP VP{\sc .tense}
%
%\ex {\bf give more precise characterization of prejacent and generalization, e.g., using logical formulas and drawing on Barker's semantics for the evaluative adjectives}
%
%Feynman was stupid to dance on the table
%
%Generalization: $stupid'(dance.on.table', f)$ ``the degree to which Feynman dancing on the table at some time is stupid is higher than the contextual standard for {\em stupid}''
%
%
%Prejacent: $dance.on.table'(f,rt)$
%
%Evaluation: $dance.on.table'(f,rt) \wedge stupid'(dance.on.table', f)$, i.e., Feynman was stupid to dance on the table
%
%\end{xlist}
\end{exe}
By (\ref{ent}), the past tense EAS in (\ref{f}) {\em Feynman was stupid to dance on the table} entails that Feynman danced on the table (the prejacent) and that, at the past reference time, the degree to which events of Feynman dancing on the table are stupid was higher than the contextual standard for {\em stupid} (the generalization). The events involved in the generalization need not be actual: for instance, it does not follow from the generalization of (\ref{f}) that an event of Feynman dancing on the table took place; this follows from the prejacent.


The proposal that the prejacent and the generalization are lexical entailments predicts that unembedded EASs are judged to be unacceptable if either of the them is false. The examples in (\ref{false}) show that that prediction is borne out. Consider (\ref{false}a), whose prejacent is false and whose generalization is true (under the assumption that the degree to which events of anybody, including Kim, being born into poverty are unfortunate is higher than the contextual standard of {\em unfortunate}): this example is correctly predicted to be unacceptable because the prejacent is false. In (\ref{false}b), on the other hand, the prejacent is true and the generalization is false (under the assumption that the degree to which events of anybody, including Sandy, being born into poverty are lucky is lower than the contextual standard of {\em lucky}). This sentence is correctly predicted to be judged to be unacceptable because the generalization is false. 

\begin{exe}
\ex\label{false}
\begin{xlist}
\ex What is true: Kim was born to rich parents
\\ \infelic Kim was unfortunate to be born into poverty.

\ex What is true: Sandy was born into poverty
\\ \infelic Sandy was lucky to be born into poverty.

\end{xlist}
\end{exe}

%The generalization of an EAS involves the meaning of the (positive) evaluative adjective, whose meaning depends on a contextual standard (e.g., \citealt{kennedy2001}). 

Having motivated that both the prejacent and the generalization are lexical entailments of EAS, we now turn to the discourse status of these contents. We argue that the discourse status of the prejacent and the generalization is not conventionally specified by the EAS: both can be entailed by the common ground at the time at which the EAS is uttered or new information at that time; in other words, neither content is associated with \citepos{brst-lang11} Strong Contextual Felicity constraint. To illustrate, consider the naturally occurring EASs in (\ref{new}). The EAS in (\ref{new}a) is acceptable in a context in which the prejacent follows from the common ground, i.e., the interlocutors already know that Trump gave Wolff unlimited access, and in which the speaker conveys the generalization as new information, i.e., that they took the degree to which events of Trump giving Wolff unlimited access are stupid to be higher than the contextual standard of {\em stupid}. The EAS in (\ref{new}b) illustrates the reverse: (\ref{new}b) is acceptable in a context in which the generalization follows from the common ground, whereas the prejacent, that MacLean grabbed Kouassi in his private parts is new information. And, finally, there are also EASs in which both the prejacent and the generalization may be new information: (\ref{new}c) is acceptable in a context in which neither the generalization nor the prejacent follow from the common ground. When a speaker utters this EAS, they may thereby be committing to the truth of the prejacent, that they bought an Xbox, as well as to the truth of the generalization, that the degree to which events of them buying an Xbox 360 elite are stupid was higher than the contextual standard of {\em stupid}.

\begin{exe}
\ex\label{new} 

\begin{xlist} 

\ex Trump was stupid to give Wolff unlimited access.\footnote{\url{https://www.eastbaytimes.com/2018/01/07/letter-trump-was-stupid-to-give-wolff-unlimited-access/}}

\ex Steven MacLean was stupid to grab Eboue Kouassi in his private parts.\footnote{\tiny{\url{https://www.heraldscotland.com/sport/17189197.brendan-rodgers-steven-maclean-was-stupid-to-grab-eboue-kouassi-in-his-private-parts/}}}

\ex I was stupid to buy the Xbox 360 elite.\footnote{\url{www.gamespot.com/forums/xbox-association-1000003/i-was-stupid-to-buy-the-xbox-360-elite-26937557}}

\end{xlist}
\end{exe}

Not only can both the prejacent and the generalization of an EAS be new information, they can also both be at-issue with respect to the Discourse Question addressed by an utterance of the EAS. Of course, given the characterization of at-issue content as addressing the Discourse Question, only one of them is at-issue in any given utterance of an EAS. We further assume that the prejacent and the generalization are the only contenders for at-issue content of an EAS, which means that exactly one of them is at-issue in any given utterance of an EAS; the other one is not-at-issue.\footnote{The subject-related entailments of EASs mentioned in footnote \ref{sre} are, by virtue of being lexically specified presuppositions, conventionally specified as not at-issue.} To illustrate, consider the examples in (\ref{ai}), in which utterances of the EAS {\em Sam/she was smart to buy one the day they went on sale} address four distinct Discourse Questions. The Discourse Questions in (\ref{ai}a) and (\ref{ai}b) are about the prejacent: both the question of whether Sam got a ticket in (\ref{ai}a) and who got a ticket in (\ref{ai}b) are answered by the prejacent of B's utterances, that Sam got a ticket.\footnote{Some native speakers of American English prefer to produce B's utterances in (\ref{ai}a) and (\ref{ai}b) with {\em enough}: {\em She was smart enough to buy one the day they went on sale}. We hypothesize that such speakers disprefer producing EASs in which the prejacent is at-issue. Given our hypothesis that the prejacent projects when it is not at-issue, we would expect such speakers to also disprefer producing NEASs like (\ref{nat}) in which the prejacent does not project. Crucially, as discussed in section \ref{s1}, even speakers who prefer to produce (\ref{ai}a) and (\ref{ai}b) with {\em enough} can retrieve the intended interpretations of the variants without {\em enough}.} Thus, in (\ref{ai}a) and (\ref{ai}b), the prejacent is at-issue and the generalization, that the degree to which events of Sam buying a ticket the day the tickets went on sale are smart was higher than the contextual standard of {\em smart} is not at-issue. The Discourse Questions in (\ref{ai}c) and (\ref{ai}d), on the other hand, are not about the prejacent:  neither the question about B's assessment of events of Sam buying a ticket they went on sale (\ref{ai}c) nor the question of whether Sam was smart to buy a ticket the day they went on sale in (\ref{ai}d) are answered by the prejacent of B's utterances; rather they are answered by the generalization. Thus, in (\ref{ai}c) and (\ref{ai}d), the generalization is at-issue and the prejacent is not at-issue.

\begin{exe}
\ex\label{ai}

\begin{xlist}
\ex
\begin{xlist}
\exi{A:} The show was sold out. Did Sam get a ticket?
\exi{B:} She was smart to buy one the day they went on sale.
\end{xlist}

\ex
\begin{xlist}
\exi{A:} There were so few tickets for the show! Who got a ticket?

\exi{B:} Sam was smart to buy one the day they went on sale. 

\end{xlist}

\ex
\begin{xlist}
\exi{A:} How do you assess Sam's buying of a ticket the day they went on sale?

\exi{B:} Sam was smart to buy one the day they went on sale.


\end{xlist}

\ex
\begin{xlist}
\exi{A:} Was Sam smart to buy a ticket for the show the day the tickets went on sale? The price would have gone down after a few days!

\exi{B:} Sam was smart to buy one the day they went on sale. The show sold out the day the tickets went on sale.

\end{xlist}

\end{xlist}
\end{exe}

We are now ready to return to our question-based projection analysis of EASs, according to which the prejacent projects when it is not at-issue with respect to the Discourse Question addressed by the EAS, and likewise for the generalization. Consider the minimal pair in (\ref{ai2}a) and (\ref{ai2}b), where B utters the same NEAS, modulo whether the subject is a pronoun or a proper name. The prejacent of this NEAS is at-issue in (\ref{ai2}a) because A's interrogative utterance, assumed to be the Discourse Question, is about the prejacent. Thus, according to our proposal, the prejacent of B's utterance in (\ref{ai2}a) is not predicted to project but the generalization, as not-at-issue content, projects. Accordingly, B's utterance is correctly predicted to mean that Sam didn't buy a ticket the day they went on sale (the prejacent does not project) and that the degree to which events of Sam buying a ticket the day they went on sale are smart was higher than the contextual standard of {\em smart} (the generalization projects). What follows is that it would have been smart for Sam to buy a ticket on the day they went on sale. In (\ref{ai2}b), on the other hand, A's interrogative utterance is about the generalization. Thus, the prejacent of B's utterance is not at-issue and, according to our proposal, is predicted to project, whereas the generalization does not project over negation. Accordingly, B's utterance is interpreted to mean that Sam bought a ticket the day they went on sale (the prejacent projects) and that the degree to which events of Sam buying a ticket the day they went on sale are smart was not higher than the contextual standard of {\em smart} (the generalization does not project). What follows is that Sam having bought a ticket on the day they went on sale wasn't smart. 


\begin{exe}
\ex\label{ai2}

\begin{xlist}
\ex
\begin{xlist}
\exi{A:} The show was sold out. Did Sam get a ticket?
\exi{B:} She wasn't smart to buy a ticket the day they went on sale. (So, she didn't go to the show.)
\end{xlist}

\ex
\begin{xlist}

\exi{A:} You keep criticizing Sam for doing some not-so-smart things. I'm not sure I agree. Can you give me an example (of something not-so-smart that Sam did)?


\exi{B:} Sure! Remember the rock show we talked about? Sam wasn't smart to buy a ticket the day they went on sale. The price went down a couple of days later.

%\exi{A:} Was Sam smart to buy a ticket for the show the day the tickets went on sale? The price would have gone down after a few days!
%
%\exi{B:} Sam wasn't smart to buy one the day they went on sale. The price indeed went down after a few days.

\end{xlist}

\end{xlist}
\end{exe}
In sum, our proposal accounts for the projectivity of the prejacent and the generalization: the more the prejacent and the generalization are taken to be not at-issue, the more highly projective they are. Given that either the prejacent or the generalization must be at-issue in any given utterance of an EAS, the interaction in their projection falls out of the analysis: the more the generalization is taken to be at-issue, the more projective the prejacent is, and the more the prejacent is taken to be at-issue, the more projective the generalization is.

In the examples we have entertained so far, the Discourse Question was realized by an interrogative utterance, thereby making clear whether the prejacent or the generalization are at-issue. However, as noted above, the Discourse Question is often implicit in naturally occurring discourse. Although previous work has identified focus marking as constraining the question that an utterance can be taken to address (and thereby influencing projection), we do not entertain focus marking here because the prosody of EASs hasn't been investigated yet (see section \ref{s6} for some remarks). Instead, the next section identifies a novel constraint on the Discourse Question of utterances, including utterances of EASs.

\subsection{At-issue content is not redundant content}\label{s23}

We propose here that one of the constraints on the Discourse Question addressed by an utterance comes from a felicity requirement that is found in different guises in the literature: an utterance of an indicative sentence is felicitous only if the sentence is informative in the context in which it is uttered. \citet[144]{groenendijk1999}, for instance, formulated this as the requirement that indicative sentences be non-redundant. We assume here that an utterance of a sentence is informative if its at-issue content is not already entailed by the common ground of the interlocutors prior to the utterance of the sentence.\footnote{Whereas at-issue content may not be redundant, not-at-issue content varies in its discourse status: some not-at-issue content, like factive presuppositions, may be new information or already entailed by the common ground, anaphoric presuppositions must be entailed by the common ground and, finally, conventional implicatures have been argued in \citealt{potts05} to be required to be new. We follow works like \citealt{potts05} and \citealt{murray2014} in assuming that conventional implicatures are always not at-issue. Thus, while both the at-issue content and conventional implicatures may be new information, they differ at-issueness.} Since content is at-issue with respect to the Discourse Question addressed by the utterance, this means that the Discourse Question cannot be about content that is entailed by the common ground because such a Discourse Question would render the utterance uninformative. What follows is that the more a particular utterance content is taken to follow from the common ground, the less likely the Discourse Question of the utterance is about that content:

\begin{exe}

\ex\label{principle} {\bf Non-redundancy principle for at-issue content} \\ The more the truth of utterance content $c$ is taken to follow from the common ground, the less likely is it that the Discourse Question of the utterance is about $c$, i.e., the less likely $c$ is at-issue.\footnote{To be clear, this principle is meant as a conditional, not a bi-conditional: it is not the case that the less likely it is that the Discourse Question is about a particular content, the more the truth of that content is taken to follow from the common ground.}

\end{exe}

The principle in (\ref{principle}) has consequences for the interpretation of EASs: the more the truth of the prejacent is taken to follow from the common ground, the less likely the Discourse Question of the EAS is about the prejacent and, therefore, the more likely the generalization is taken to be at-issue and the less projective the generalization is. And vice versa: the more the truth of the generalization is taken to follow from the common ground, the less likely the Discourse Question of the EAS is about the generalization and, therefore, the more likely the prejacent is taken to be at-issue and the less projective the prejacent is. To illustrate, consider the naturally occurring examples in (\ref{god}) and (\ref{nat}b): the prejacent of (\ref{god}) is highly projective according to the ratings obtained in the corpus study, and that of (\ref{nat}b) is not, as discussed in section \ref{s1}.

\begin{exe}

\ex\label{god} God offers Hope to Hispanics! In His pages are solutions to every immigration problem. God loves citizens and immigrants equally. His solutions are for all of us. They are practical. They work. He is not stupid to think so.

\exi{(\ref{nat}b)}  Now I knew someone was in the junkyard and the cold wind was
carrying the cries. I wasn't stupid to go stumbling through the
junkyard in the dark and get hurt.

\end{exe}
The prejacent of the NEAS in (\ref{god}) is that God thinks that his solutions are practical and work. The strength of the inference from the common ground to the truth of this prejacent is quite high. Thus, the principle in (\ref{principle}) predicts that the Discourse Question of (\ref{god}) is not likely to be about the prejacent, which is therefore more likely to be taken to be not-at-issue and to be highly projective, as observed. Now consider the NEAS in (\ref{nat}b), whose generalization is that the degree to which events of the speaker stumbling through the junkyard in the dark and getting hurt are stupid was higher than the contextual standard of {\em stupid}. The strength of the inference from the common ground to the truth of the generalization is quite high. Thus, the principle in (\ref{principle}) predicts that the Discourse Question of (\ref{nat}b) is not likely to be about the generalization; rather, the Discourse Question is expected to be about the prejacent, which, as predicted, is observed to not project. 

\subsection{Interim summary and predictions}\label{s24}

This section developed a question-based projection analysis for EASs that builds on previous analyses in assuming that the more likely utterance content is taken to be not at-issue,  the more projective it is (\citealt{best-question,brst-ar,tbd-variability}). To apply this analysis to EASs, section \ref{s22} motivated that both the prejacent and the generalization are lexical entailments of EASs that can but need not be at-issue. In contrast to analyses according to which the prejacent is lexically specified as a presupposition (e.g., \citealt{barker02,oshima09b}), the analysis developed in this section predicts that both the prejacent and the generalization can project and, furthermore, that when the prejacent projects, the generalization does not, and vice versa. The analysis also incorporates a novel constraint on Discourse Questions, the `Non-redundancy principle for at-issue content'. 

As discussed in section \ref{s1}, the analysis developed in \citealt{karttunen-etal2014} captures the projectivity of the prejacent and the generalization, as well as the interaction in their projection, by assuming two lexical entries for each evaluative adjective: one according to which the prejacent projects and the generalization does not, and one according to which the generalization projects and the prejacent does not. There are three reasons why we think our analysis compares favorably to that presented in \citealt{karttunen-etal2014}. First, our analysis is more parsimonious: whereas both analyses build on prior analyses of projective content (i.e., do not introduce additional machinery),  evaluative adjectives are required to be ambiguous under \citepos{karttunen-etal2014} analysis but not ours. Second, the interaction between the prejacent and the generalization is stipulated in \citealt{karttunen-etal2014}, but falls out of the independently-motivated assumption of our analysis that only one utterance content is at-issue. Finally, as shown by the discussion in section \ref{s21}, the fact that projective content can but need not project is generally not attributed to lexical ambiguity, neither under analyses that take projective content to be conventionally specified nor on analyses that derive projection differently. Thus, against the broader landscape of projection analyses, \citepos{karttunen-etal2014} analysis is more idiosyncratic than ours.

The next section provides experimental evidence in support of our analysis. Specifically, the two experiments manipulated the extent to which the generalization was taken to be at-issue and investigated the following two predictions of the analysis:

\begin{exe}
\ex\label{pred} The more likely the generalization is taken to be at-issue,
\begin{xlist}
\ex \ldots the more projective the prejacent is. (Experiment 1)

\ex \ldots the less likely the prejacent is taken to be at-issue. (Experiment 2)

\end{xlist}
\end{exe}
To investigate the predictions in (\ref{pred}), we measured the projectivity of the prejacent (Experiment 1) and the at-issueness of the prejacent (Experiment 2).

\section{Empirical evidence for the question-based projection analysis}\label{s3}

According to the question-based projection analysis developed in section \ref{s2}, the prejacent and the generalization of EASs are predicted to project to the extent that they are taken to be not at-issue. Preliminary evidence for this prediction comes from \citepos{tbd-variability} finding that at-issueness predicts projection for 19 projective contents. One of the contents they investigated was the prejacent of EASs with {\em stupid} embedded under the polar question operator. The four EASs investigated are given in (\ref{stupid2}):

\begin{exe}
\ex\label{stupid2}

\begin{xlist}

\ex Was Raul stupid to cheat on his wife?

\ex Were John's kids stupid to be in the garage?

\ex Is Mary's daughter stupid to be biting her nails?

\ex Is Richie stupid to be a stuntman? \hfill (\citealt{tbd-variability}, Appendix A)

\end{xlist}

\end{exe}
\citet{tbd-variability} collected projection and at-issueness ratings on a scale: Figure \ref{f-corr} shows participants' projectivity ratings for the prejacents of the four items in (\ref{stupid2}) against their not-at-issueness ratings. There is a clear relationship between at-issueness and projectivity: the more a participant rated a prejacent as not-at-issue, the more projective they rated it ($r =$ .91; when not collapsing over the four questions $r =$ .57). 

\begin{figure}[h!]
\centering

\includegraphics[width=.95\textwidth]{figures/Exp1a-subject-projai-stupid}

\caption{Projectivity ratings against not-at-issueness ratings for the prejacents of four EASs in \citealt{tbd-variability}. Each dot represents one participant's ratings. Linear smoothers with 95\% confidence intervals overlaid.}
\label{f-corr}
\end{figure}
\citepos{tbd-variability} finding provides preliminary evidence that the projectivity of the prejacent of {\em stupid} is sensitive to its at-issueness. However, in \citepos{tbd-variability} experiment, at-issueness was only measured, not manipulated, and only the prejacents of EASs with {\em stupid} were investigated. In the experiments reported on in this section, we provide more direct evidence for the question-based projection analysis by manipulating at-issueness and including a wide variety of evaluative adjectives and items.

Further preliminary evidence for our analysis comes from an experiment reported on in \citealt{karttunen-etal2014}, which investigated the projection of the prejacents of NEASs with 19 evaluative adjectives ({\em arrogant, brave, careless, cruel, evil, foolish, fortunate, heroic, humble, lucky, mean, nice, polite, rude, sensible, smart, stupid, sweet, wise}). The materials included one triple for each evaluative adjective, like the triple for {\em wise} in (\ref{paul}). The experiment manipulated whether there was a predisposition to a content that is related to our generalization. Specifically, each triple included a NEAS referred to by \citet{karttunen-etal2014} as `consonant', which means that ``there is a predisposition to assume or grant that for NP to VP would be Adj'' (p.237). For instance, the NEAS in (\ref{paul}a) is consonant because for Paul to take the best piece is smart; this is comparable to what we have characterized as the generalization following from the common ground. Each triple also included a NEAS that \citet{karttunen-etal2014} referred to as `dissonant', which means that ``there is a predisposition to assume or grant that for NP to VP would not be Adj'' ({\em ibid}). The NEAS in (\ref{paul}c) is dissonant because for the man to take the worst piece is not smart; this is comparable to the falsity of the generalization following from the common ground. Finally, the third NEAS in each triple was considered `neutral', i.e., neither consonant or dissonant, like (\ref{paul}b).

\begin{exe}
\ex\label{paul} Sample stimuli from \citealt[241]{karttunen-etal2014}
\begin{xlist}
\ex Paul wasn't smart to take the best piece. \hfill [consonant]
\ex Sally wasn't smart to take the middle piece.  \hfill [neutral]
\ex The man wasn't smart to take the worst piece. \hfill [dissonant]
\end{xlist}
\end{exe}

Karttunen and his colleagues found that the prejacent of dissonant NEASs was more likely to project than the prejacent of neutral ones, and that the prejacent of neutral ones was more likely to project than that of consonant ones. These findings support the prediction of our analysis, that the discourse status of the generalization influences the projectivity of the prejacent. There are, however, some concerns with their experiment. First, the experiment included only one triple for each evaluative adjective, and so the findings potentially have limited generalizability. Second, the triples were not normed to establish that native speakers of American English share Karttunen and his colleagues' assumptions about consonance and dissonance. Relatedly, the stimuli were
presented to the participants without a context even though context can influence whether the generalization follows from the common ground. For instance, if Sally is on a diet and {\em middle piece} in (\ref{paul}b) is understood as {\em middle piece of the cake}, then (\ref{paul}b) is not neutral, but dissonant: for Sally to take the middle piece (or any piece) is not smart given that she is on a diet. To address these concerns, our Experiment 1 included a greater variety of items, which were normed with native speakers of American English, and presented to the participants with a context.\footnote{\label{f-git}The
materials, data and the R code for generating the figures and analyses of the experiments reported on in this paper are available at {\tt [name of GitHub repository]}.} 
%\url{https://github.com/judith-tonhauser/evaluative-adjectives}.} 

\subsection{Experiment 1: Projectivity of the prejacent}\label{s31}

The experiment we report on in this section investigated the prediction in (\ref{pred}a), that the more likely the generalization is taken to be at-issue, the more projective the prejacent is. The at-issueness of the generalization was manipulated by having the truth of the generalization be more likely to follow from the common ground in one condition than the other. We collected gradient projection ratings for prejacents in the two conditions.

\subsubsection{Methods}

\paragraph{Participants} 152 participants with US IP addresses and at least 97\% of HITs approved were recruited on Amazon's Mechanical Turk platform (ages: 18-69, mean age: 33). They were paid 45 cents.

\paragraph{Materials} Stimuli consisted of two-sentence discourses. In the target stimuli, the first sentence was a context sentence and the second sentence was a NEAS with one of the following 10 evaluative adjectives: {\em stupid, smart, wise, fortunate, lucky, brave, polite, mean, foolish} and {\em rude}. For each adjective, there were 6 pairs of target stimuli, for a total of 60 pairs of target stimuli. We used a 2 x 2 within-participant design: one factor was whether the truth of the generalization was taken to follow from the common ground (levels: less likely vs.\ more likely); the other factor was whether the content of the NEAS or the context sentence determined the likeliness with which the truth of the generalization was taken to follow from the common ground (levels: content vs.\ context). The sample stimuli in (\ref{cont}) and (\ref{context}) illustrate these two factors. For a pair of stimuli in the Content condition, shown in (\ref{cont}), the context sentences were identical and the two NEASs differed in how likely the truth of the generalization is taken to follow from the common ground. In (\ref{cont}a), the generalization is that the degree to which events of Sally losing her wallet are fortunate was higher than the contextual standard of {\em fortunate} (to be clear: the generalization is established on the EAS, not the NEAS). Here, it is less likely that the truth of the generalization is taken to follow from the common ground. In (\ref{cont}b), the generalization is that the degree to which events of Sue speaking French are fortunate was higher than the contextual standard of {\em fortunate}. Here, it is more likely that the truth of the generalization is taken to follow from the common ground. For pairs of stimuli in the Context condition, shown in (\ref{context}), the NEASs were identical and the differing context sentences determined how likely the truth of the generalization is taken to follow from the common ground. In (\ref{context}a), the generalization is that the degree to which events of Jane prank-calling the police are smart was higher than the contextual standard of {\em smart}; here, it is less likely that the truth of the generalization is taken to follow from the common ground.  In (\ref{context}b), the generalization is that the degree to which events of Jane calling the police Jane are smart was higher than the contextual standard of {\em smart}; here,  it is more likely that the truth of the generalization is taken to follow from the common ground. In both the Content and Context conditions, we expected the projectivity of the prejacent to be higher when the truth of the generalization was less likely to follow from the common ground, i.e., when the generalization is more likely to be at-issue.

\begin{exe}
\ex\label{cont} Sample stimuli in the Content condition 

\begin{xlist}
\ex Sue was traveling in France. She wasn't fortunate to lose her wallet.  \hfill [generalization less likely]

\ex Sue was traveling in France. She wasn't fortunate to speak some French. \hfill [gen.\ more likely]
\end{xlist}

\ex\label{context} Sample stimuli in the Context condition

\begin{xlist}
\ex Jane was prank-calling people. She wasn't smart to call the police. \hfill [generalization less likely]

\ex Jane saw a man with a gun. She wasn't smart to call the police. \hfill [generalization more likely]
\end{xlist}
\end{exe}
To assess the projectivity of the prejacent, participants were asked about the truth of the prejacent, e.g., `Did Sue lose her wallet?' in (\ref{cont}a), `Did Sue speak some French?' in (\ref{cont}b), and `Did Jane call the police?' in (\ref{context}a) and (\ref{context}b).\footnote{This diagnostic for projectivity differs from the `certain that' diagnostic used in the web-based corpus study or in \citealt{tbd-variability}. Experiment 1 relied on a different diagnostic because it was run before the `certain that' diagnostic was developed. Under the assumption that participants only take the prejacent to be true if it follows from the two-sentence discourse, i.e., if the author of the two-sentence discourse is committed to the prejacent, we take the two diagnostics to be comparable in diagnosing projectivity. See \citealt{tbd-variability} for a discussion of additional diagnostics for projectivity.}

Of the 6 pairs of target stimuli per evaluative adjective, 3 pairs were in the Content condition and 3 in the Context condition. We selected these 60 pairs of target stimuli from a total of 120 pairs of potential target stimuli for which we collected ratings in a norming study from native speakers of American English about whether the truth of the generalization was taken to follow from the common ground. For each evaluative adjective, we chose the 3 pairs of target stimuli in the Content and Context conditions such that the generalization was most likely to be taken to follow from the common ground in one member of the pair and least likely to follow in the other member of the pair. The norming study is described in detail in Appendix \ref{a-norming} and the full set of stimuli is provided in Appendix \ref{a-Exp1}.

The 120 target stimuli were distributed across 12 lists of 10 target stimuli so that each evaluative adjective occurred once per list. Each list included 5 target stimuli in which the generalization is more likely to be taken to follow from the common ground and 5 in which the generalization is less likely to be taken to follow from the common ground. Each list included 5 stimuli from the Content condition and 5 from the Context condition. To assess whether participants were attending to the task, the same 6 control stimuli were added to each list, for a total of 16 stimuli per list (see Appendix \ref{a-Exp1} for the control stimuli). 

\paragraph{Procedure}

Participants were told that they would read short descriptions of scenarios and were asked a question about each scenario. They were randomly assigned to a list and presented with the 16 stimuli, one after the other, in random order. As shown in Figure \ref{f-trial-exp1}, they gave their response to the polar question on a 7-point Likert scale labeled at four points: No/1, Possibly
no/3, Possibly yes/5, Yes/7. 

\begin{figure}[h!]
\centering

\fbox{\includegraphics[width=.7\paperwidth]{figures/trial-exp1}}

\caption{A sample trial in Experiment 1}\label{f-trial-exp1}
\end{figure}

After rating the 16 stimuli, participants completed a brief questionnaire about their age, their
native language(s) and, if English is a native language, whether it is
American English, as opposed to e.g., Indian or Australian English.
Participants were told that they would be paid no matter how they
responded to these questions, in order to encourage them to answer
truthfully.


\paragraph{Data exclusion} The data from 5 participants who did not self-identify as native speakers of American English were excluded. 13 participants were excluded based on their responses to the control stimuli, leaving data from 134 participants (ages: 18-69; mean age: 33).

\subsubsection{Results and discussion}

Each of the 120 target stimuli received between 9 and 14 ratings (mean: 11.2). Figure \ref{f-condis} shows the mean projectivity ratings for prejacents of NEASs in the Content condition (left panel) and the Context condition (right panel) by how likely the truth of the generalization was taken to follow from the common ground. As expected, prejacents were more projective when the truth of the generalization was less likely to follow from the common ground than when it was more likely to follow from the common ground: in the Content condition, the mean projectivity ratings were 5.2 and 2.6, respectively; in the Context condition, the mean projectivity ratings were 4.7 and 2.9, respectively. As shown by the overlaid adjective means in Figure \ref{f-condis}, the effect of the discourse status of the generalization was observed for all of the adjectives, albeit to varying degrees. These findings provide empirical support for the prediction that at-issueness influences projection with EASs: when the truth of the generalization is less likely to follow from the common ground, the generalization is more likely to be at-issue, in which case the prejacent is more likely to be not at-issue and hence more projective.

\begin{figure}[H]
\begin{center}
\includegraphics[scale=.75]{../exp2-projection/graphs/mean-certainty-ratings}

\caption{Mean projectivity rating for prejacents of NEASs in the Content condition (left panel) and the Context condition (right panel) by how likely the truth of the generalization is taken to follow from the common ground. Error bars indicate bootstrapped 95\% confidence intervals. Adjective means in the two conditions overlaid.}\label{f-condis}
\end{center}
\end{figure}

We fitted ordinal mixed-effects regression models to the target data in the Content and Context conditions (668 and 672 data points, respectively), using the {\tt clmm} function of the {\tt ordinal} package (\citealt{Christensen2013}) in {\em R} (\citealt{r}; version 3.2.0). The models predicted projectivity ratings on the 7-point Likert scale from the fixed effect of the discourse status of the generalization (with `more likely' as the reference level). The models included the maximal random effects structure justified by the data and the theoretical assumptions: random by-participant intercepts (capturing differences in projectivity between participants) and random by-item intercepts (capturing differences in projectivity between context/adjective/{\em to-}infinitive combinations) as well as random slopes for the discourse status of the generalization by participant  (capturing that the effect of the discourse status may vary across participants). We obtained $p$-values by comparing models via likelihood ratio tests.

There was a significant main effect of the discourse status of the generalization such that the prejacent of items in which the truth of the generalization is less likely to follow from the common ground received higher projectivity ratings in the Content ($\beta$ = 3.29, $SE$ = 0.34, $z$ = 9.6, $LR(1)$ = 73, $p <$ .001) and Context ($\beta$ = 2.1, $SE$ = 0.37, $z$ = 5.73, $LR(1)$ = 29.68, $p <$ .001) conditions. These findings suggest that the discourse status of the generalization influences the projectivity of the prejacent, as predicted by the analysis developed in section \ref{s2}: prejacents of NEASs in which the truth of the generalization is less likely to follow from the common ground, i.e., is more likely to be at-issue, are more projective than prejacents of NEASs in which the truth of the generalization is more likely to follow from the common ground, i.e., more likely to be not at-issue. Our findings also suggest that readers attend both to information from the EAS and to information from the context in determining the extent to which the truth of the generalization follows from the common ground and, therefore, the projectivity of the prejacent.

\subsection{Experiment 2: At-issueness of the prejacent}\label{s32}

This experiment tested the prediction in (\ref{pred}b), that the more likely the generalization is taken to be at-issue, the less likely the prejacent is taken to be at issue. The at-issueness diagnostic used in Experiment 2 relies on the assumption that at-issue and not-at-issue content differ in the extent to which it is up for debate and can be directly assented/dissented with. For previous uses of diagnostics that rely on this assumption see, e.g., \citealt{amaral-etal07,xue-onea11,murray2014,anderbois-etal2015,destruel-etal2015,tonhauser-sula6,syrett-koev2015} and \citealt{tbd-variability}. The diagnostic we used in Experiment 2 is the same as in  \citepos{tbd-variability} Exps.~2. The 3-turn dialogue in (\ref{sure}) illustrates how the diagnostic was set up on the basis of the appositive content associated with nominal appositives. The speaker of the first turn, Debby, utters an indicative sentence with the target expression, here a nominal appositive, and thereby commits herself to various utterance contents, including the appositive content that Martha's new car is a BMW. The speaker of the second turn, Harry, utters the question {\em Are you sure?}, thereby challenging some content of Debby's utterance. In the third turn, Debby utters an indicative sentence in which the content to be diagnosed for at-issueness, here the appositive content of her first utterance, realizes the content of the  complement of {\em sure}, thereby identifying it as the content that Debby took the Harry to be challenging. 

\begin{exe}
\ex\label{sure} At-issueness diagnostic from \citealt{tbd-variability}
\begin{xlist}
\exi{Debby:} Martha's new car, a BMW, was expensive.

\exi{Harry:} Are you sure?
	
\exi{Debby:} Yes, I am sure that Martha's new car is a BMW. 
\end{xlist}
\end{exe}
To assess whether the relevant content is up for debate, i.e., at-issue, participants were asked whether Debby's utterance answered Harry's question, with `yes' and `no' as response options. A `yes' response was taken to indicate that the relevant content was at-issue: the content was targeted by Harry's question and, therefore, Debby answered Harry's question. A `no' response, in turn, was taken to indicate that the relevant content was not at-issue: the content was not targeted by Harry's question and, therefore, Debby did not answer Harry's question. We assume that the more `yes' responses a content receives, the more likely it is taken to be at-issue.

\subsubsection{Methods}

\paragraph{Participants} 75 participants with US IP addresses and at least 97\% of HITs approved were recruited on Amazon's Mechanical Turk platform (18-66, mean: 35). They were paid 35 cents.

\paragraph{Materials} Stimuli consisted of 3-turn discourses between Debby and Harry, as shown in (\ref{sure2}) and (\ref{sure22}). In the target stimuli, the first turn of each discourse consisted of a past tense EAS that realized one of the 10 evaluative adjectives explored in Experiment 1. The second turn of the target stimuli consisted of Harry asking {\em Are you sure?}. In the third turn, Debby uttered {\em Yes, I am sure that}, with the prejacent of the EAS realized as the content of the complement of {\em sure that}. There were 6 stimuli for each of the 10 evaluative adjectives, for a total of 60 target stimuli. To manipulate the at-issueness of the generalization, in 3 of the stimuli for each adjective the truth of the generalization was more likely to be taken to follow from the common ground, as in (\ref{sure2}); in the other 3, it was less likely to be taken follow from the common ground, as in (\ref{sure22}). Given the `Non-redundancy principle for at-issue content' in (\ref{principle}), this means that the generalization of (\ref{sure2}) is less likely to be taken to be at-issue than that of (\ref{sure22}).  The full set of stimuli is in Appendix \ref{a-Exp2}. 

\begin{exe}
\ex\label{sure2} Truth of the generalization is more likely to follow from the common ground
\begin{xlist}
\exi{Debby:} Jane was stupid to post her social security number on Facebook. 

\exi{Harry:} Are you sure?

\exi{Debby:} Yes, I am sure that Jane posted her social security number on Facebook.
\end{xlist}

\ex\label{sure22} Truth of the generalization is less likely to follow from the common ground
\begin{xlist}
\exi{Debby:} Jane was stupid to dance like that. 

\exi{Harry:} Are you sure?

\exi{Debby:} Yes, I am sure that Jane danced like that.
\end{xlist}
\end{exe}

Participants were asked whether Debby answered Harry's question: a `yes' response was taken to indicate that Harry's question targeted the prejacent, i.e., the prejacent of Debby's utterance was taken to be at-issue; a `no' response was taken to indicate that Harry's question did not target the prejacent, i.e., the prejacent of Debby's utterance was not taken to be at-issue. We expected the prejacent of EASs for which the truth of the generalization was more likely to follow from the common ground, i.e., was more likely to be not at-issue, to be more at-issue than the prejacent of EASs for which the truth of the generalization was less likely to follow from the common ground.

The 60 target stimuli were distributed across 6 lists so that each of the 10 adjectives occurred once per list. Each list had 5 target stimuli for which the generalization was more likely to be taken to follow from the common ground and 5 target stimuli for which the generalization was less likely to be taken to follow from the common ground. To assess whether participants were attending to the task, each list also included the same two control stimuli. We also included on each list the same four stimuli that assessed the at-issueness of other projective content (see Appendix \ref{a-Exp2}). In sum, each of the 6 lists consisted of 16 stimuli.

\paragraph{Procedure} 

Participants were told that they were at a party and that, upon walking into the kitchen, they overhear a short dialogue between Debby, the party host, and another guest, Harry. They were randomly assigned to a list and presented with the 16 stimuli, one after the other, in random order. They gave their ratings to the question of whether Debby answered Harry's question with two radio buttons labeled `yes' and `no', as shown in Figure \ref{f-trial-exp2}.  

\begin{figure}[H]
\centering

\fbox{\includegraphics[width=10cm]{figures/trial-exp2}}

\caption{A sample trial in Experiment 2}\label{f-trial-exp2}
\end{figure}

After rating the 16 stimuli, participants filled out a brief questionnaire about their age, their native language(s) and, if English is a native language, whether it is
American English, as opposed to e.g., Indian or Australian English.
Participants were told that they would be paid no matter how they
responded to these questions, in order to encourage them to answer
truthfully.

\paragraph{Data exclusion} The data from 2 participants who did not self-identify as native speakers of American English were excluded. 5 participants answered `no' to at least one of the two control stimuli. Their data were excluded, leaving data from 68 participants (ages: 18-66; mean: 35).

\subsubsection{Results and discussion}

Each of the 60 target stimuli received between 10 and 14 ratings (mean: 11.3).  Figure \ref{f-ai-by-adj} shows the proportion of `yes' responses, indicating at-issueness, in the two conditions:  as expected, the prejacent of EASs for which the truth of the generalization is more likely to follow from the common ground received more `yes' responses than the prejacents of EASs for which the truth of the generalization is less likely to follow from the common ground. This finding supports the prediction of our analysis that the prejacent of EASs for which the generalization is not at-issue are more at-issue than the prejacent of EASs for which the generalization is at-issue. As shown by the overlaid adjective means in Figure \ref{f-ai-by-adj}, the effect of the at-issueness of the generalization on the at-issueness of the prejacent was observed to varying degrees for all of the adjectives except {\em brave}.

\begin{figure}[H]
\begin{center}
\includegraphics[scale=.85]{../exp3-at-issueness/graphs/proportion-ai-by-condition}

\caption{Proportion of `yes' responses, indicating at-issueness of the prejacent, by condition. Error bars indicate bootstrapped 95\% confidence intervals. Adjective means in the two conditions overlaid.}\label{f-ai-by-adj}
\end{center}
\end{figure}

To statistically evaluate the effect of the at-issueness of the generalization on the at-issueness of the prejacent, we fitted a Bayesian binomial mixed effects model with weakly informative priors using the R package \verb|brms| \citep{buerkner2017}.\footnote{We fit a Bayesian binomial mixed effects model rather than a frequentist mixed effects model because this allowed us to fit a model with a full random structure that would not converge with frequentist
methods \citep{Nicenboim2016}. In fact, when we fit regular binomial mixed effects models, predicting response from a fixed effect of the discourse status of the generalization, the models only converged if we included either random by-item intercepts or random by-participant intercepts, but not both. Qualitatively, the results were identical to those obtained via the Bayesian method.} The model predicted the log odds of `yes' over `no' ratings from a fixed effect of how likely the truth of the generalization was taken to follow from the common ground (with `less likely' as the reference level). We included the maximal random effects structure justified by the design: random intercepts for item (capturing random differences in at-issueness between items) and participant (capturing random differences in at-issueness between participants) as well as random by-participant slopes for the at-issueness of the generalization (capturing that the effect of the at-issueness of the generalization may vary by participant). Four chains converged after 2000 iterations each (warmup = 1000, \(\hat{R}=1\) for all estimated parameters).

In order to evaluate the evidence for an effect of the at-issueness of the generalization, we report 95\% credible intervals and the posterior probability $P(\beta > 0)$ that the at-issueness coefficient $\beta$ is greater than zero. A 95\% credible interval (CI) demarcates the range of values that comprise 95\% of probability mass of our posterior beliefs such that no value inside the CI has a lower probability than any point outside it \citep{Jaynes1976, Morey2016}. There is substantial evidence for an effect if zero is (by a reasonably clear margin) not included in the 95\% CI and $P(\beta > 0)$ is close to zero or one. Posterior probabilities tell us the probability that the parameter has a certain value, given the data and model (these probabilities are not frequentist $p$-values). In order to present statistics as close to widely used frequentist practices, and following \citealt{Nicenboim2016}, we defined an inferential criterion that seems familiar (95\%), but the strength of evidence should not be taken as having clear cut-off points (such as in a null-hypothesis significance testing framework).

The model provided evidence for the predicted effect of the at-issueness of the generalization on the at-issueness of the prejacent, such that the prejacent of EASs for which the truth of the generalizations was more likely to follow from the common ground were more likely to receive a `yes' (at-issue) rating than the prejacent of EAS for which the truth of the generalization was less likely to follow from the common ground  (posterior mean $\beta$ = 1.29, 95\% CI={[}0.69,1.87{]}, $P(\beta > 0)$ = 1). This finding suggests that the prejacent of EASs is more likely to be taken to be at-issue when the generalization is less likely to be taken to be at-issue than when the generalization is more likely to be taken to be at-issue, as predicted by the analysis developed in section \ref{s2}.
  
\subsection{Summary}

According to the analysis of EASs developed in section \ref{s2}, the prejacent is not a lexically specified presupposition but projects to the extent that it is taken to be not at-issue with respect to the Discourse Question addressed by the utterance. The analysis predicts that the more likely the generalization is taken to be at-issue, the more likely the prejacent is taken to be not-at-issue and projective. In this section, we provided empirical evidence for two predictions of this analysis by manipulating how likely the truth of the generalization was taken to follow from the common ground, i.e., how likely the generalization is at-issue. Experiment 1 showed that the prejacent is more projective when the generalization is more likely to be at-issue than when the generalization is less likely to be at-issue. Experiment 2 showed that the prejacent is more at-issue when the generalization is less likely to be at-issue than when the generalization is more likely to be at-issue. In sum, the two experiments provide evidence in support of the question-based analysis of the projection and at-issueness of the two contents as well as their interaction.



\section{Interspeaker variation}\label{s5}

Before concluding the paper, we briefly return to the interspeaker variation we noted in section \ref{s1}: not all native speakers of American English would produce NEASs in which the prejacent does not project; many speakers prefer variants with {\em enough}. What might this variation be due to? 

One hypothesis is that speakers differ in their lexical entries for evaluative adjectives. The acceptability rating study presented in Appendix \ref{s-acc} found that about 20-30\% of the 94 self-reported native speakers of American English judged NEASs in which the prejacent does not project to be acceptable. Let's assume that such speakers have a lexical entry for evaluative adjectives according to which the prejacent is not lexically specified as a presupposition (as in the analysis developed in section \ref{s2}); this would allow them to produce NEASs in which the prejacent projects as well as ones in which it doesn't project. And let's assume that the remaining speakers (about 70-80\%) are ones that would not produce NEASs in which the prejacent does not project. We can hypothesize that they have a lexical entry according to which the prejacent is lexically specified as a presupposition, thereby resulting in them preferring to produce EASs in which the prejacent is not-at-issue and therefore projects. When interpreting NEASs, speakers in this second group should consistently assign projecting interpretations to prejacents of NEASs in which the truth of the generalization is not likely to be taken to follow from the common ground: not only is the projecting interpretation the lexically specified one, but it is also supported by the common ground. A similar hypothesis can be found in \citet[243]{karttunen-etal2014}, who suggested that there are about 3 times as many speakers who prefer giving interpretations to NEASs in which the prejacent projects, using the lexical entry in (\ref{lex}a), than speakers who prefer giving interpretations to NEASs in which the prejacent doesn't project, using the lexical entry in (\ref{lex}b). Thus, both hypotheses lead to the expectation that about 70-80\% of native speakers of American English consistently assign projecting interpretations to NEASs in which the truth of the generalization is not likely to be taken to follow from the common ground.

A post-hoc analysis of the findings of Experiment  1 suggests that both hypotheses should be rejected. Recall that each of the 134 participants in that experiment rated the projectivity of the prejacent of NEASs on a 7-point Likert scale: they rated the projectivity of the prejacent of 5 NEASs for which the truth of the generalization is more likely to follow from the common ground and of 5 NEASs for which the truth of the generalization is less likely to follow from the common ground. To explore the aforementioned hypotheses, we calculated each participants' mean projectivity ratings for these two types of NEAS: these mean projectivity ratings are an indication of how projective the 134 participants rated the prejacent of the two types of NEAS. Under both hypotheses, we expect 70-80\%  of these participants to assign highly projective interpretations to the prejacent of NEASs for which the truth of the generalization is less likely to follow from the common ground. 

The histogram in Figure \ref{f-dialect} shows the 134 participants' mean projectivity ratings for the two types of NEASs. The left panel reveals that, of the 134 participants, only 23 (17\%) had mean projectivity ratings of at least 6.5 for the NEASs in which the negation of the generalization follows from the common ground; when considering mean projectivity ratings of at least 5.5, this number still only rises to 55 (41\%) participants. Thus, we do not find that a majority of the 134 participants assigned highly projective interpretations to prejacents of NEASs for which the truth of the generalization does not follow from the common ground. This observation calls into question the hypothesis that a majority of native speakers of American English have a lexical entry for evaluative adjectives according to which the prejacent is a lexically-specified presupposition. 

\begin{figure}[h!]
\begin{center}
\includegraphics[scale=.9]{/Users/tonhauser.1/Documents/current-research-topics/NSF-NAI/prop-att-experiments/2evaluative-adjectives/content-context-experiment/results/graphs/count-of-participants}

\caption{Histogram of participants in Experiment 1 by their mean projectivity ratings and by whether the truth of the generalization is less likely (left panel) or more likely (right panel) to be taken to follow from the common ground}\label{f-dialect}
\end{center}
\end{figure}

We leave further explorations of the question of how to capture the variability observed among native speakers of American English to future research.

\section{Conclusions}\label{s6}

Over the past decades, several different types of analyses have been proposed for different classes of projective content: on some analyses, projection is derived from the lexical specification of content as a presupposition (e.g., \citealt{heim83,vds92}) or from conventional specification as a conventional implicature (e.g., \citealt{potts05,murray2014}); on other analyses, projection is derived from the lexical specification of alternatives, together with pragmatic reasoning (e.g., \citealt{abusch02,abusch10,romoli2015}); on yet other analyses, projection is derived from discourse status (e.g., \citealt{abrusan2011,abrusan2013,abrusan2016,brst-salt10,best-question,brst-ar}). Which type of analysis is empirically adequate for any given projective content depends on empirical properties of that content (for discussion see, e.g., \citealt{kadmon01,potts05,brst-lang11,tbd-variability,tonhauser-etal-sub23}). In this paper, we argued for an analysis that derives the projection of the prejacent and the generalization of EASs from their discourse status based on two properties: first, the prejacent of EASs is not highly projective, in line with other content whose projection has been derived from its discourse status; second, there is an interaction between the projection of the prejacent and the generalization that falls out from the independently-motivated assumption that exactly one utterance content is at-issue. While it is possible to capture the projectivity of the prejacent and the generalization as well as their interaction from  lexical specification of content as presupposed, we argued in section \ref{s24} that our question-based analysis more parsimonious, less stipulative and less idiosyncratic. Thus, the research presented here shows once again that the observation that content is projective is insufficient grounds for settling on a particular analysis; rather, additional properties of the content must be considered, such as how projective it is, whether it is associated with a Strong Contextual Felicity constraint (\citealt{brst-lang11}), and which aspects of discourse its projection depends on. Rather than assuming by default that content projects because it is a lexically specified presupposition, future research must consider the growing landscape of analyses of projective content and the properties of such content that identify which analysis is empirically adequate.

There are at least two questions about EASs that should be addressed in future work. The first concerns the interaction between the projection of the prejacent and the generalization. As discussed in section \ref{s1}, this interaction has not yet been observed for other projective content. Why does it arise with EASs, but not, for instance, for utterances of sentence with factive or change-of-state predicates? We can only speculate here that it may have to do with the prejacent and the generalization of EASs being independent of one another in that neither is a precondition for the truth of the other. To illustrate, consider (\ref{f}) {\em Feynman was stupid to dance on the table}. For the generalization to be true, it does not matter whether Feynman danced on the table; likewise, for the prejacent to be true, it does not matter whether the degree to which events of Feynman dancing on the table are stupid was higher than the contextual standard of stupid. In being independent of one another, the lexical entailments of EASs differ from the lexical entailments of sentences with factive or change-of-state predicates, like {\em Sam knows that the meeting was canceled} or {\em Sam stopped going to church}: in order for Sam to know that the meeting was canceled, the content of the complement must be true, and in order for Sam to not go to church anymore, the pre-state content, that they previously went to church, must be true. Future research needs to establish whether this is the reason for the interaction between the projection of the prejacent and the generalization.

A second question that future research should address is how information structure constrains the projection of the prejacent and the generalization. As mentioned in section \ref{s21}, previous research has established that information structure, in particular prosodically marked focus, provides a cue to the questions addressed by utterances of sentences with other projective content. We hypothesize that prosodically marked focus will also provide a cue to the question addressed by EASs and, hence, to which content projects. 


%\end{document}

\appendix

\section{Acceptability rating experiment}\label{s-acc}

The acceptability rating experiment was designed to explore the acceptability of  NEASs whose prejacent does not project. 

\subsection{Methods}

\paragraph{Participants} 134 participants with US IP addresses and at least 97\% of HITs approved were recruited on Amazon's Mechanical Turk platform (ages: 20-72; mean: 33). They were paid 45 cents.

\paragraph{Materials} Stimuli consisted of three-sentence discourses. In the target stimuli, the last sentence was either a NEAS, as in (\ref{jenny}a), or a variant of the NEAS in which the evaluative adjective was followed by {\em enough}, as in (\ref{jenny}b). The first two sentences of the discourses denied the truth of the prejacent of the third sentence to ensure that the prejacent of the third sentence could not project: in (\ref{jenny}), for instance, the first two sentences convey that Jenny left the baby unattended, which contradicts the prejacent, that Jenny kept an eye on the baby at all times. We hypothesized that participants would judge the NEAS to be acceptable only if they would realize an interpretation in which the prejacent projects with the NEAS. We further hypothesized that the variant with {\em enough} would be judged to be acceptable by all participants.

\begin{exe} 
\ex\label{jenny} 
\begin{xlist}
\ex Jenny was baby-sitting for her sister. She
left the baby unattended. \\ \underline{She wasn't smart to keep an eye on the baby at all times.} 

\ex Jenny was baby-sitting for her sister. She
left the baby unattended. \\ \underline{She wasn't smart enough to keep an eye on the baby at all times.} 

\end{xlist}
\end{exe} 

The target stimuli consisted of 6 pairs of three-sentence discourses like (\ref{jenny}) for each of the 10 evaluative adjectives explored in the experiments reported on in the main body of the paper, for a total of 60 pairs of target stimuli. The full set of target stimuli is provided in the GitHub repository (url removed for anonymity). The 120 target stimuli were distributed across 12 lists of 10 stimuli each so that each adjective occurred only once per list, and each list included 5 NEASs and 5 variants with {\em enough}. 

To assess whether participants were paying attention to the task, the same 8 control stimuli were added to each list, for a total of 18 stimuli per list. The control stimuli consisted of three-sentence discourses in
which the last sentence contradicted the second one, as illustrated in
(\ref{claire}). We expected the last sentence of the control stimuli to be judged to be unacceptable as part of the discourse.

\begin{exe} 
\ex\label{claire} 
\begin{xlist}
\ex Claire fell down the stairs. She broke a leg and some ribs. \uline{She was glad to not be hurt.} 

\ex Earl had a bad toothache. He called his dentist. \uline{He forgot to call his dentist.}

\ex Ross was doing his laundry. He found coins in his pocket. \uline{He didn't manage to find coins anywhere.}

\ex Charles was on a date with Susanne. He was having a bad time. \uline{He was happy to be having a great time.}

\ex Keith was hiking up a steep mountain. He didn't make it to the top. \uline{He was ecstatic to have made it to the top.}

\ex Liv went on a trip to Europe. She spent a lot of time in Italy. \uline{She failed to visit Italy.}

\ex Fran lived in Paris. She shared an apartment with some friends. \uline{She wasn't sad to live alone.}

\ex Tess had failed many college classes. She didn't graduate. \uline{She was proud to have graduated from college.}

\end{xlist}
\end{exe}


\paragraph{Procedure} Participants were told that they would read descriptions of a scenario and asked to judge whether the last, underlined sentence sounds good as part of the description. They gave their ratings on a 7-point Likert scale
labeled at four points: No/1, Possibly no/3, Possibly yes/5,
Yes/7). Participants were randomly assigned to a list and presented with the 18 stimuli, one after the other, in a random order. After rating the 18 stimuli, participants filled out a brief questionnaire about their age, their native language(s) and, if English is a native language, whether it is American English, as opposed to e.g., Indian or Australian English.
Participants were told that they would be paid no matter how they
responded to these questions, in order to encourage them to answer
truthfully.


\paragraph{Data exclusion} 

The ratings from 2 participants who did not self-identify as native speakers of American English were excluded. 38 participants gave a rating higher than Possibly no/3 to one or more of the eight unacceptable control stimuli. The ratings from these participants were also excluded, leaving data from 94 participants (ages: 20-67; mean: 33). 

\subsection{Results}

As expected, target stimuli with {\em enough} were judged to be acceptable: the mean rating for such items was 5.9 and most ratings were 5 or higher (409 of 477 responses,
86\%). Ratings for the target stimuli with NEASs were lower, with a mean rating of 4.2. As shown in the left panel of Figure \ref{f-acc}, NEASs received lower ratings overall.

To identify whether a participant judged NEASs in which the prejacent does not project to be acceptable, we calculated each participants' mean rating of the 5 stimuli with a NEAS. We then subtracted this mean rating from their mean rating for the 5 stimuli with {\em enough}. The resulting `acceptability score' is a measure of how acceptable NEASs with non-projecting prejacents are compared to acceptable stimuli with {\em enough}. An acceptability score of 0 means that the participant judged the NEASs with non-projecting prejacents as acceptable as stimuli with {\em enough}. The right panel of Figure \ref{f-acc} plots the participants' acceptability scores; the red line indicates an acceptability score of 0. Of the 94 participants, 20 (21\%) had an acceptability score of at least 0 and 27 (29\%) had an acceptability score of at least -0.5. Thus, although there are also many participants who do not judge NEASs in which the prejacent does not project to be acceptable, a sizable portion of (self-reported) native speakers of American English judge such sentences to be acceptable.

\begin{figure}[h!]

\hspace*{-.5cm}\includegraphics[scale=.8]{../acceptability-rating-study/graphs/histogram} \includegraphics[scale=.8]{../acceptability-rating-study/graphs/acceptability-rating-difference}


\caption{Count of ratings of target stimuli (left panel) and acceptability scores of participants (right panel)}\label{f-acc}

\end{figure}

\section{Corpus-based web study: Projectivity of the prejacent}\label{a-corpus}

\citet{karttunen-etal2014} provided naturally occurring examples that showed that the prejacent of NEASs need not project. The goal of the study we present in this section was to more systematically explore the projection of the prejacent in naturally occurring NEAS. To explore the projectivity of the prejacent, native speakers of American English were presented with naturally occurring NEAS and asked to rate the projectivity of the prejacent.

\subsection{Methods}

\paragraph{Participants} We recruited 260 participants with US IP addresses and at least 97\% of prior HITs approved on Amazon's Mechanical Turk platform (ages: 18-83; mean: 35). They were paid 75 cents.  


\paragraph{Materials} Using the online interface of the EnTenTen corpus,\footnote{The EnTenTen 2012 corpus has 11,191,860,036 words
(\url{www.sketchengine.co.uk}, \citealt{ententen}).} we searched for NEASs that matched one of the strings in (\ref{string}), where ADJ was one of the following 10 evaluative adjectives that have been analyzed as presupposing the prejacent (e.g., \citealt{norrick78,karttunen-etal2014}): {\em stupid, smart, wise,
fortunate, lucky, brave, polite, mean, foolish} and {\em rude}.

\begin{exe}
\ex\label{string} {\tt \{am not / are not / aren't / is not / isn't / was not / wasn't / were not / \\ weren't / will not be / won't be\}} ADJ {\tt to}
\end{exe}
We limited ourselves to examples from the American English part of the EnTenTen corpus (region: Am) and to examples with referential subjects. We did not find examples for the adjectives {\em polite, brave} and {\em rude}. For adjectives with more than 10 pages of results, we extracted all relevant examples from the first 10 pages and then extracted a random selection of examples from the remaining pages. The final set of target stimuli consisted of 59 NEAS with preceding and following context sentences that the first author judged to be relevant to understanding the NEAS (29 with past tense, 27 with non-past tense and 3 with {\em will}).

Each target stimulus was attributed to an author who was identified by name, as shown for the sample stimuli in (\ref{target1}). As shown in the `Question to participants' presented with each stimulus in (\ref{target1}), we used the `certain that' diagnostic for projectivity to assess whether participants took the author to be committed to the prejacent, i.e., whether the prejacent projects over negation. For other applications of the `certain that' diagnostic for projection see \citealt{tonhauser-salt26,djaerv-bacovcin-salt27,stevens-etal2017} and \citealt{tbd-variability}. In these `Questions to participants', the prejacent was realized as the finite complement clause of {\em certain}. The tense of the finite clause that realized the prejacent was determined by the temporal and aspectual properties of the
NEAS: it was past tense for past tense NEAS, as in (\ref{target1}a), and future tense for future tense NEAS, as in (\ref{target1}b); for non-past tense NEAS, the finite clause that realized the prejacent was realized in the non-past tense if the eventuality denoted by the prejacent was
stative, as in (\ref{target1}c), and a disjunction of a past tense and a future tense verb if the prejacent was eventive, as in (\ref{target1}d).


\begin{exe} \ex\label{target1} Sample target stimuli and questions to
participants 

\begin{xlist} 

\ex Shawn: Mr. Anderson --  Just discovered your
site on The 1939-40 New York World's Fair. It brought back a lot of
memories for me. Thanks for the time you spent in constructing this
site. I was not fortunate to visit the 1939-40 World's Fair but I
had an uncle who did.
\\ Question to participants: Is Shawn certain that he visited the 1939-40 World's Fair?

\ex Megan: For sure P1 will not be stupid to create a WiMAX netbook to allow YTL or AMAX customers to use.
\\ Question to participants: Is Megan certain that P1 will create a WiMAX netbook to allow YTL or AMAX customers to use?

\ex Frank: God offers Hope to Hispanics! In His pages are solutions to every immigration problem. God loves citizens and immigrants equally. His solutions are for all of us. They are practical. They work. He is not stupid to think so.
\\ Question to participants: Is Frank certain that God thinks that his solutions work?

\ex Anna: The flight attendants are not wise to invoke the specter
of a strike. \\ Question to participants: Is Anna certain that the
flight attendants invoked or will invoke the specter of a strike?

\end{xlist}
\end{exe}

We created 8 lists of 8 target stimuli each and distributed the 59 target stimuli across the 8 lists; 5 target stimuli occurred in two lists. Each list also included the following
two control stimuli which were used to assess whether
participants were attending to the task, for a total of 10 stimuli per list. For the control stimulus in (\ref{cntrl}a), we expected participants to not take the speaker to be committed to the relevant content (that Earl called the dentist) and, for the control stimulus in (\ref{cntrl}b), we expected participants to take the speaker to be committed to the relevant content (that Tess crossed the finish line).

\begin{exe}
\ex\label{cntrl} Control stimuli
\begin{xlist}
\ex Jack: Earl had a toothache. He forgot to call his dentist. \\ Question to participants: Is Jack certain that Earl called his dentist?

\ex  David: Tess participated in a marathon. She was happy to cross
the finish line. \\ Question to participants: Is David certain that Tess
crossed the finish line? \end{xlist} \end{exe}

\paragraph{Procedure} Participants were told that they would read short snippets coming from various internet blogs and forums. They were randomly assigned to a list and presented with the 10 stimuli, one after the other, in random order. As shown in \ref{f-corpus-trial}, they gave their ratings to the `certain that' question on a 7-point Likert scale labeled at four points: no/1, possibly no/3, possibly yes/5, yes/7.

\begin{figure}[h!]
\begin{center}
\fbox{\includegraphics[width=10cm]{figures/f-corpus-trial}}
\end{center}
\caption{A sample trial in Experiment 1}\label{f-corpus-trial}
\end{figure}

After responding to the 10 stimuli, 
participants filled out a brief questionnaire about their age, their
native language(s) and, if English is a native language, whether it is
American English, as opposed to e.g., Indian or Australian English.
Participants were told that they would be paid no matter how they
responded to these questions, in order to encourage them to answer
truthfully. 

\paragraph{Data exclusion} The responses from 6 participants who did not self-identify as native speakers of American English were excluded. 28 participants gave a rating higher than 3 to the control stimulus in (\ref{cntrl}a), for which we hypothesized that participants would not take the speaker to be committed to the relevant content, or a rating lower than 5 to the control stimulus in (\ref{cntrl}b), for which we hypothesized that participants would take the speaker to be committed to the relevant content. These responses suggest that these participants were not attending to the task or interpreted the task differently. We therefore also excluded the data from these 28 participants, leaving data from 226 participants (ages: 18-83; mean: 35).

\subsection{Results}

There were 26-32 certainty ratings for each of the 59 NEAS (mean: 28.5), except for the five NEASs that appeared on two lists, for which there were 52-58 certainty ratings each (mean: 54). We calculated the mean certainty rating of the prejacent of each NEAS: the lowest mean certainty rating was 1.2, the highest was 6.7 and the mean certainty rating overall was 3.2. Given that participants gave their ratings on a 7-point scale, with the lowest rating (1) indicating that the speaker was taken to not be certain about the relevant content and the highest rating (7) indicating that the speaker was taken to be certain about the relevant content, this mean certainty rating of 3.2 already suggests that the prejacent of naturally occurring NEAS is not highly projective. Figure \ref{f-corpus} in section \ref{s2} showed the mean certainty rating of each NEAS by evaluative adjective. The distribution of mean certainty ratings again suggests that the prejacent of naturally occurring NEASs is projective, but not highly so, and that NEASs in which the prejacent does not project are well-attested in naturally occurring data. 
%
%In (\ref{stupid}) to (\ref{smart}), we provide naturally occurring NEASs with a highly projective prejacent (a.-examples) and with a prejacent whose projectivity is very low (b.-examples); the bold-faced number included
%with each example is the mean certainty rating of the prejacent.\footnote{\citet{karttunen2013}
%assumed that NEAS with the  adjectives {\em
%lucky, fortunate, unlucky} and {\em unfortunate} in the non-future
%tenses only give rise to non-projecting interpretations; projecting
%interpretations were attributed to meta-linguistic negation. Since
%examples like (\ref{fort}a) cannot reasonably be attributed to
%meta-linguistic negation, we continue to treat these adjectives as being compatible with 
%both projecting and non-projecting interpretations (see also \citealt{karttunen-etal2014}).} Importantly, the a.- and b.-examples in (\ref{stupid}) to (\ref{smart}) differ not only in whether the prejacent projects over negation, but also in the evaluation: in the a.-examples, the evaluation is negated and in the b.-examples it is counterfactual. In (\ref{stupid}a), for instance, the evaluation is that it is not the case that God is stupid to think that his solutions work. In (\ref{stupid}b), on the other hand, the evaluation is that the Google founders would have been stupid to take on that liability.
% 
%
%
%%Problematic result
%%\ex Zidan was once again the goalscorer for Mainz as he scored their only goal in the 81st minute. The Egyptian also had two chances in the first half but wasn't lucky to score. [4.5]
%
%\begin{exe}
%
%
%\ex\label{stupid} {\em stupid}
%
%\begin{xlist}
%
%\ex God offers Hope to Hispanics! In His pages are solutions to every immigration problem. God loves citizens and immigrants equally. His solutions are for all of us. They are practical. They work. He is not stupid to think so. \hfill {\bf [6.3]}
%
%\ex There is no chance a true large scale museum can be built. The federal government cannot take on the humongous legal liability for such a hazardous site. The Google founders are not stupid to take on that liability either. \hfill {\bf [1.9]}
%
%\end{xlist}
%
%\ex\label{fort} {\em fortunate}
%
%\begin{xlist}
%
%\ex Four months after her last Prolotherapy treatment  Tessie is still
%doing great. But what would have happened if she was not
%fortunate to be married to Joe? \hfill {\bf [6.4]}
%
%\ex\label{fair}  Mr. Anderson  Just discovered your site on The 1939-40 New York World's Fair. It brought back a lot of memories for me. Thanks for the time you spent in constructing this site. I was not fortunate to visit the 1939-40 World's Fair but I had an uncle who did \hfill {\bf [1.2]}
%
%\end{xlist}
%
%\ex\label{lucky} {\em lucky}
%
%\begin{xlist}
%
%\ex Not even a 40 foot ladder could reach this roof  so I had to use a boom (bucket) lift. I got a 60' lift for this particular job  and it worked out nicely. I took the above photo while working in the lift. You can see the tiles. They look nice and neat in the photo  but up close  they were riddled with gaps. I spent a great deal of time sealing them. I was lucky to have level and relatively unobstructed ground. I was not lucky to have recent rains and a several ton machine to drive on the soft ground. \hfill {\bf [5.9]}
%
%%\ex  It seems that there is not a specific test kit to certificate thin client  or  if existing  i was not \fbox{lucky} to find it. \hfill {\bf [1.5]}
%
%\ex Alice: Good day. My husband and i were married for four years now but before that he already has a child with his ex gf. We are not lucky to have a child of our own. \hfill {\bf [1.4]}
%
%\end{xlist}
%
%
%\ex {\em foolish}
%
%\begin{xlist} \ex Out east in St. Bernard Parish Chad Blanchard is
%hoping he was not foolish to spend his savings on reopening
%Charlie's restaurant -- leaving himself with just 15 dollar bills in his
%back pocket. \\ \hspace*{.1cm} \hfill {\bf [5.4]}
%
%%\ex Archbishop Williams categorically pointed out that none of the people he had trained to become pastors could rub shoulders with him for the reason that he did not impart all of his knowledge on them. He emphasized that he was not \fbox{foolish} to give all that he had as a celebrated man of God to the people he trained  because he knew for a fact that when he did that the people could one day rub shoulders with him. \hfill {\bf [2.6]}
%
%\ex Olmert is a dinosaur politician  he will exist for a long time in politics  and therefore will not be foolish to make the same mistake twice. \hfill {\bf [2.1]}
%
%\end{xlist}
%
%\ex\label{smart} {\em smart}
%
%\begin{xlist}
%
%\ex I think Mark Cuban was not smart to let a successful coach go in Avery Johnson. \hfill {\bf [6.7]}
%
%%\ex You and I must trust God for our finances  but that's no license to
%%spend sillily. You and I ought to trust God for safety in the car  but
%%we are not \fbox{smart} to pass on a blind curve. \hfill {\bf [2.5]}
%
%\ex Once you installed this free Notipage webpage monitoring software
%then just make your first alert. It's worth mentioning here that the app
%will not be smart to keep an eye on web pages that are only easy
%to get to by logging in to a site or service.   \hfill {\bf [2.2]}
%
%\end{xlist}
%
%\end{exe}


\section{Norming study for Exp.~1}\label{a-norming}

The goal of the norming study was to identify generalizations of EASs such that their truth is more or less likely to follow from the common ground. 

\subsection{Methods}

\paragraph{Participants} 270 participants, with US IP addresses and at least 97\% of HITs approved, were recruited on Amazon's Mechanical Turk platform (ages 18-68, 1 undeclared; mean: 32). They were paid 45 cents.


\paragraph{Materials} For each of the ten evaluative adjectives of Exp.~1, we created 12 pairs of two-sentence stimuli such that the first sentence was a context sentence and the second sentence a NEAS, as in (\ref{cont}) and (\ref{context}). We hypothesized, for each pair of NEASs, that the truth of the generalization of one member of the pair is more likely to follow from the common ground and less likely for the other member. Target stimuli in the norming study consisted of two-sentence discourses where the first sentence was the context sentence of the stimuli in Exp.~1 and the second sentence was the prejacent of the NEAS, as shown in (\ref{cont-norm}) and (\ref{context-norm}) for the stimuli in (\ref{cont}) and (\ref{context}), respectively. Given that there were 120 pairs of potential stimuli for Exp.~1, there were 120 pairs of two-sentence discourses in the norming study, i.e., a total of 240 two sentence discourses. The full set of target stimuli is provided in the GitHub repository mentioned in footnote \ref{f-git}. To assess whether the truth of the generalization  follows from the common ground, participants were asked to assess the truth of the corresponding generalization. The generalizations were presented in the past tense to cohere with the tense of the discourses that participants read.


\begin{exe}
\ex\label{cont-norm} Sample stimuli in the Content condition 

\begin{xlist}
\ex Sue was traveling in France. She lost her wallet. \\ (Question to  participants: `Was Jane fortunate to lose her wallet?')

\ex Sue was traveling in France. She spoke some French. \\ (Question to participants: `Was Jane fortunate to speak French?')

\end{xlist}

\ex\label{context-norm} Sample stimuli in the Context condition

\begin{xlist}
\ex Jane was prank-calling people. She called the police. \\ (Question to participants: `Was Jane smart to call the police?')

\ex Jane saw a man with a gun. She called the police. \\ (Question to participants: `Was Jane smart to call the police?')
\end{xlist}
\end{exe}

The 240 target stimuli were distributed into 24 lists of 10 target stimuli so that each list included one target stimulus for each evaluative adjective. Each list included 5 target stimuli in which the truth of the generalization was hypothesized to be likely to follow from the common ground and 5 in which in which the truth of the generalization was hypothesized to be unlikely to follow from the common ground. Each list included 5 stimuli each from the target stimuli created for the Context and Content conditions.

To assess whether participants were attending to the task, the same 4 control stimuli were added to each list, for a total of 14 stimuli per list. The control stimuli consisted of two-sentence discourses, as shown in (\ref{controls-norming}): the two controls in (\ref{controls-norming}a-b) were expected to receive a positive answer and the two controls in (\ref{controls-norming}c-d) were expected to receive a negative answer.

\begin{exe}
\ex\label{controls-norming}
\begin{xlist}
\ex Earl worked in London last year. He was a teacher at a private school. 
\\ (Is it true that Earl worked at a private school in London?)

\ex Liv was having a birthday party. She bought a cake for her party.
\\ (Is it true that Liv bought a cake for her birthday party?)

\ex Claire was knitting a sweater. She was using red yarn. 
\\ (Is it true that Claire was knitting a blue sweater?)

\ex Ross graduated from college yesterday. He was an excellent student. 
\\ (Is it true that Ross dropped out of college?)

\end{xlist}
\end{exe}

\paragraph{Procedure}

Participants were told that they would read descriptions of a scenario and asked to respond to a question about the scenario. They gave their responses on a 7-point Likert scale labeled at four points: No/1, Possibly no/3, Possibly yes/5,
Yes/7). Participants were randomly assigned to a list and presented with the 14 stimuli, one after the other, in a random order. After rating the 14 stimuli, participants filled out a brief questionnaire about their age, their native language(s) and, if English is a native language, whether it is American English, as opposed to e.g., Indian or Australian English. Participants were told that they would be paid no matter how they
responded to these questions, in order to encourage them to answer
truthfully.


\paragraph{Data exclusion} 

The ratings from 9 participants who did not self-identify as native speakers of American English were excluded. 31 participants gave ratings lower than Possibly yes/5 to the control stimuli in (\ref{controls-norming}a-b), for which we expected the speaker to be taken to be committed to the relevant content, or higher than Possibly no/3 to the control stimuli in (\ref{controls-norming}c-d), for which we expected the speaker to not be taken to be committed to the relevant content. The ratings from these participants were also excluded, leaving data from 230 participants (ages: 18-68; mean: 32). 

\subsection{Analysis}

Each of the 240 target discourses received between 7 and 13 ratings (mean: 9.6). As shown in Table \ref{t-mean-norming1}, stimuli for which the truth of the generalization was hypothesized to be likely to follow generally received high ratings and stimuli for which this was not the case generally received low ratings.

\begin{table}[!h]
\centering

\begin{tabular}{l|cc}
& \multicolumn{2}{c}{Truth of generalization follows from common ground} \\
& more likely & less likely \\\hline

Content condition &  6.2 (1.2) &  1.6 (1.1) \\
Context condition &  6.0 (1.3) &  2.2 (1.6) \\

\hline
\end{tabular}

\caption{Mean ratings and standard deviations for the 240 stimuli in the four conditions}\label{t-mean-norming1}

\end{table}

Of the 120 pairs of target stimuli (60 pairs in the Context and Content conditions each), we selected the 6 best pairs for each evaluative adjective: 3 in the Context condition and 3 in the Content condition: the `best' pairs of stimuli were those where the mean ratings of the two members of the pair were as high and low as possible, respectively. The mean ratings and standard deviations for the 120 stimuli selected for Exp.~1 are shown in Table \ref{t-mean-final}.

\begin{table}[!h]
\centering

\begin{tabular}{l|cc}
& \multicolumn{2}{c}{Truth of the generalization follows from common ground:} \\
& more likely & less likely \\\hline

Content condition &  6.5 (0.9) &  1.3 (0.8) \\
Context condition &  6.1 (1.2) &  1.7 (1.2) \\

\hline
\end{tabular}

\caption{Mean ratings and standard deviations for the 120 selected stimuli in the four conditions}\label{t-mean-final}

\end{table}

%
%The best  pairs of stimuli in the content condition have the following
%properties: the mean for the consonant member of the pair is 5.75 or
%higher and the mean for the dissonant member of the pair is 2.1 or
%lower. The difference between the means of the pairs was 5.2 (sd: 0.5)).
%
%The best pairs of stimuli in the context condition have the following
%properties, with five exceptions, discussed below: the mean for the
%consonant member of the pair is 5.2 or higher and the mean for the
%dissonant member of the pair is 2.3 or lower. For three adjectives, we
%had to select pairs of less ideal stimuli (since no better stimuli were
%available), namely for {\em brave, lucky} (2 pairs each) and {\em mean} (1 pair). The difference between the means of the selected pairs was 4.4 (sd: 1).
%


\section{Materials of Experiment 1}\label{a-Exp1}

The control stimuli of Experiment 1 also consisted of two-sentence discourses: the content of interest in (\ref{control-exp1}a-c) was hypothesized to be true, and the content of interest in (\ref{control-exp1}d-f) was hypothesized to be false. Participants were excluded if they gave a response lower than Possibly yes/5 to at least one of the control stimuli in (\ref{control-exp1}a-c) or higher than Possibly no/3 to at least one of the control stimuli in (\ref{control-exp1}d-f). 

\begin{exe}
\ex\label{control-exp1} 

\begin{xlist}

\ex Fran lived in Paris. She wasn't sad to live alone. \\ (Question to participants: Did Liv live alone?)

\ex Charles served in Iraq. He was proud to have been a soldier. \\ (Question to participants: Was Charles a soldier?)

\ex Tess participated in a marathon. She was happy to cross the finish line. \\ (Question to participants: Did Tess cross the finish line?)


\ex Earl had a bad toothache. He forgot to
call his dentist. \\ (Question to participants: Did Earl call his
dentist?)

\ex Ross was doing his laundry. He didn't manage to find coins anywhere. \\ (Question to participants: Did Ross find coins anywhere?)

\ex Liv went on a trip to Europe. She failed to visit Italy. \\
(Question to participants: Did Liv visit Italy?)

\end{xlist}
\end{exe}


There were 12 target stimuli for each of the 10 evaluative adjectives: 6 stimuli in the Content condition (Cn) and 6 in the Context condition (Cx). For stimuli marked whose coding ends with an `T' (i.e., [xxT]), the norming study established that the strength of the inference to the truth of the generalization from the common ground was high, i.e., the generalization follows from the common ground. For stimuli whose coding ends with an `F' (i.e., [xxF]), the norming study established that the strength of the inference to the falsity of the generalization was high.

\begin{itemize}[leftmargin=15pt,itemsep=-1pt]

\item {\bf brave}

\begin{enumerate}[leftmargin=15pt,topsep=0pt,itemsep=0pt]

\item[CnT] 	Greg wrote a controversial article.	He	wasn't brave	to use his own name.
\item[CnT] 	Greg saw a child drowning in a river.	He	wasn't brave	to jump into the river.
\item[CnT] 	Greg saw a man hit a dog.	He	wasn't brave	to stand up to the man.
\item[CnF] 	Greg wrote a controversial article.	He	wasn't brave	to use a fake name.
\item[CnF] 	Greg saw a man hit a dog.	He	wasn't brave	to walk away.
\item[CnF] 	Greg saw a child drowning in a river.	He	wasn't brave	to watch from the river bank.
\item[CxT] 	Greg was being attacked by five men.	He	wasn't brave	to take them on.
\item[CxT] 	Greg was asked to speak up against the crime boss.	He	wasn't brave	to agree to do it.
\item[CxT] 	Greg was offered a job as a lion tamer.	He	wasn't brave	to take the job.
\item[CxF] 	Greg was asked to help his friend move.	He	wasn't brave	to agree to do it.
\item[CxF] 	Greg was asked to mow his neighbor's lawn.	He	wasn't brave	to take the job.
\item[CxF] 	Greg was being tickled by two small boys.	He	wasn't brave	to take them on.

\end{enumerate}

\item {\bf foolish}

\begin{enumerate}[leftmargin=15pt,topsep=0pt,itemsep=0pt]


\item[CnT] 	Kate's doctor told her to lose weight.	She	wasn't foolish	to ignore him.
\item[CnT] 	Kate took her nephew to the playground.	She	wasn't foolish	to wear high heels.
\item[CnT] 	Kate was working a very stressful job.	She	wasn't foolish	to ignore her health problems.
\item[CnF] 	Kate's doctor told her to lose weight.	She	wasn't foolish	to start exercising.
\item[CnF] 	Kate took her nephew to the playground.	She	wasn't foolish	to bring water and a snack.
\item[CnF] 	Kate was working a very stressful job.	She	wasn't foolish	to take relaxation classes.
\item[CxT] 	Kate had a broken ankle.	She	wasn't foolish	to go for a run.
\item[CxT] 	Kate was completely broke.	She	wasn't foolish	to go shopping.
\item[CxT] 	Kate was drunk at a bar.	She	wasn't foolish	to take her top off.
\item[CxF] 	Kate needed a new dress.	She	wasn't foolish	to go shopping.
\item[CxF] 	Kate was getting a breast exam.	She	wasn't foolish	to take her top off.
\item[CxF] 	Kate wanted to exercise.	She	wasn't foolish	to go for a run.

\end{enumerate}

\item {\bf fortunate}

\begin{enumerate}[leftmargin=15pt,topsep=0pt,itemsep=0pt]



\item[CnT] 	Sue was traveling in France.	She	wasn't fortunate	to speak some French.
\item[CnT] 	Sue was bitten by a shark.	She	wasn't fortunate	to get away with a few cuts.
\item[CnT] 	Sue was training to become a dancer.	She	wasn't fortunate	to have a great sense of rhythm.
\item[CnF] 	Sue was bitten by a shark.	She	wasn't fortunate	to lose a leg.
\item[CnF] 	Sue was traveling in France.	She	wasn't fortunate	to get robbed.
\item[CnF] 	Sue was training to become a dancer.	She	wasn't fortunate	to develop back problems.
\item[CxT] 	Sue needed to get some rest.	She	wasn't fortunate	to fall asleep.
\item[CxT] 	Sue spontaneously went to the beach.	She	wasn't fortunate	to be wearing flip-flops.
\item[CxT] 	Sue needed a present for her sick friend.	She	wasn't fortunate	to have a flower bouquet.
\item[CxF] 	Sue is allergic to pollen.	She	wasn't fortunate	to have a flower bouquet.
\item[CxF] 	Sue was at her best friend's wedding.	She	wasn't fortunate	to fall asleep.
\item[CxF] 	Sue spontaneously went on a hike in the mountains.	She	wasn't fortunate	to be wearing flip-flops.

\end{enumerate}

\item {\bf lucky}

\begin{enumerate}[leftmargin=15pt,topsep=0pt,itemsep=0pt]
\item[CnT] 	Eve bought a raffle ticket. 	She	wasn't lucky	to win the first prize.
\item[CnT] 	Eve was in a car accident.	She	wasn't lucky	to get away unscathed.
\item[CnT] 	Eve wanted to be a model.	She	wasn't lucky	to have beautiful skin.
\item[CnF] 	Eve wanted to be a model.	She	wasn't lucky	to have bad skin.
\item[CnF] 	Eve bought a raffle ticket. 	She	wasn't lucky	to lose it the next day.
\item[CnF] 	Eve was in a car accident.	She	wasn't lucky	to break her neck.
\item[CxT] 	Eve has a 5th grade education.	She	wasn't lucky	to get a minimum wage job.
\item[CxT] 	Eve was a finalist in The Bachelor.	She	wasn't lucky	to be chosen.
\item[CxT] 	Eve did not understand her chemistry class.	She	wasn't lucky	to get a B.
\item[CxF] 	Eve was the best student in this class.	She	wasn't lucky	to get a B.
\item[CxF] 	Eve was called in for jury duty.	She	wasn't lucky	to be chosen.
\item[CxF] 	Eve has a first rate college education.	She	wasn't lucky	to get a minimum wage job.

\end{enumerate}

\item {\bf mean}

\begin{enumerate}[leftmargin=15pt,topsep=0pt,itemsep=0pt]



\item[CnT] 	Jack bumped his shopping cart into a woman.	He	wasn't mean	to laugh when she cried.
\item[CnT] 	Jack walked past an old man with a cane.	He	wasn't mean	to push the man.
\item[CnT] 	Jack saw a hungry dog.	He	wasn't mean	to pretend to have food for him.
\item[CnF] 	Jack walked past an old man with a cane.	He	wasn't mean	to help him across the street.
\item[CnF] 	Jack bumped his shopping cart into a woman.	He	wasn't mean	to apologize to her.
\item[CnF] 	Jack saw a hungry dog.	He	wasn't mean	to feed him a can of food.
\item[CxT] 	Jack didn't like the movie his wife was watching.	He	wasn't mean	to turn off the movie.
\item[CxT] 	Jack's daughter is lactose intolerant.	He	wasn't mean	to give her a milk shake.
\item[CxT] 	Jack's wife wanted to sleep in.	He	wasn't mean	to wake her up at 5am.
\item[CxF] 	Jack's wife had asked him to wake her really early.	He	wasn't mean	to wake her up at 5am.
\item[CxF] 	Jack's young daughter was watching an R-rated movie.	He	wasn't mean	to turn off the movie.
\item[CxF] 	Jack's daughter was a little hungry.	He	wasn't mean	to give her a milk shake.

\end{enumerate}


\item {\bf polite}

\begin{enumerate}[leftmargin=15pt,topsep=0pt,itemsep=0pt]

\item[CnT] 	Chad had insulted his wife.	He	wasn't polite	to apologize.
\item[CnT] 	Chad was standing in front of his friend's door.	He	wasn't polite	to knock gently.
\item[CnT] 	Chad was visiting his grandmother.	He	wasn't polite	to bring a gift.
\item[CnF] 	Chad was visiting his grandmother.	He	wasn't polite	to insult her caretaker.
\item[CnF] 	Chad was standing in front of his friend's door.	He	wasn't polite	to be eavesdropping.
\item[CnF] 	Chad had insulted his wife.	He	wasn't polite	to laugh at her tears.
\item[CxT] 	Chad's friend wanted to change her clothing.	He	wasn't polite	to close his eyes.
\item[CxT] 	Chad was watching a theater play.	He	wasn't polite	to applaud.
\item[CxT] 	Chad watched a street comedian.	He	wasn't polite	to laugh.
\item[CxF] 	Chad was in a meeting with his boss.	He	wasn't polite	to close his eyes.
\item[CxF] 	Chad saw an old lady trip on the street.	He	wasn't polite	to laugh.
\item[CxF] 	Chad saw an old lady trip on the street.	He	wasn't polite	to applaud.

\end{enumerate}

\item {\bf rude}

\begin{enumerate}[leftmargin=15pt,topsep=0pt,itemsep=0pt]



\item[CnT] 	Ann was standing in front of her friend's door.	She	wasn't rude	to be eavesdropping.
\item[CnT] 	Ann had insulted her husband.	She	wasn't rude	to laugh at his tears.
\item[CnT] 	Ann was visiting her older brother.	She	wasn't rude	to insult his wife.
\item[CnF] 	Ann was visiting her older brother.	She	wasn't rude	to bring a gift.
\item[CnF] 	Ann was standing in front of her friend's door.	She	wasn't rude	to knock gently.
\item[CnF] 	Ann had insulted her husband.	She	wasn't rude	to apologize.
\item[CxT] 	Ann got reprimanded by her boss.	She	wasn't rude	to ignore him.
\item[CxT] 	Ann was eating pasta with her friends.	She	wasn't rude	to use her fingers.
\item[CxT] 	Ann's neighbor greeted her.	She	wasn't rude	to ignore him.
\item[CxF] 	Ann untied her shoes.	She	wasn't rude	to use her fingers.
\item[CxF] 	Ann's neighbor made an inappropriate comment.	She	wasn't rude	to ignore him.
\item[CxF] 	Ann's boyfriend made fun of her haircut.	She	wasn't rude	to ignore him.

\end{enumerate}


\item {\bf smart}

\begin{enumerate}[leftmargin=15pt,topsep=0pt,itemsep=0pt]


\item[CnT] 	Jane was baby-sitting for her sister.	She	wasn't smart	to keep an eye on the baby at all times.
\item[CnT] 	Jane's computer was hacked.	She	wasn't smart	to change her passwords immediately.
\item[CnT] 	Jane wanted to get a good job.	She	wasn't smart	to get her high school degree.
\item[CnF] 	Jane's computer was hacked.	She	wasn't smart	to keep using the same passwords.
\item[CnF] 	Jane wanted to get a good job.	She	wasn't smart	to drop out of high school.
\item[CnF] 	Jane was baby-sitting for her sister.	She	wasn't smart	to leave the baby unattended.
\item[CxT] 	Jane saw a man with a gun.	She	wasn't smart	to call the police.
\item[CxT] 	Jane's father couldn't hear the TV.	She	wasn't smart	to turn up the volume.
\item[CxT] 	Jane wanted to get a good job.	She	wasn't smart	to go to school.
\item[CxF] 	Jane was prank-calling people.	She	wasn't smart	to call the police.
\item[CxF] 	Jane's neighbor complained about the loud music.	She	wasn't smart	to turn up the volume.
\item[CxF] 	Jane had the measles.	She	wasn't smart	to go to school.


\end{enumerate}


\item {\bf stupid}

\begin{enumerate}[leftmargin=15pt,topsep=0pt,itemsep=0pt]

\item[CnT] 	Zack left the bar drunk.	He	wasn't stupid	to drive home.
\item[CnT] 	Zack was offered some contaminated heroin.	He	wasn't stupid	to inject it.
\item[CnT] 	Zack discovered that his girlfriend was cheating.	He	wasn't stupid	to marry her.
\item[CnF] 	Zack was offered some contaminated heroin.	He	wasn't stupid	to refuse to take it.
\item[CnF] 	Zack left the bar drunk.	He	wasn't stupid	to call a taxi.
\item[CnF] 	Zack discovered that his girlfriend was cheating.	He	wasn't stupid	to break up with her.

\item[CxT] 	Zack had the measles.	He	wasn't stupid	to go to school.
\item[CxT] 	Zack saw two wasps in his drink.	He	wasn't stupid	to take a sip.
\item[CxT] 	Zack was sitting in the bath tub.	He	wasn't stupid	to use the hair dryer.
\item[CxF] 	Zack was drinking a glass of wine.	He	wasn't stupid	to take a sip.
\item[CxF] 	Zack had wet hair.	He	wasn't stupid	to use the hair dryer.
\item[CxF] 	Zack wanted to get a good job.	He	wasn't stupid	to go to school.

\end{enumerate}

\item {\bf wise}

\begin{enumerate}[leftmargin=15pt,topsep=0pt,itemsep=0pt]


\item[CnT] 	Paul ran a marathon on Sunday.	He	wasn't wise	to go to bed early the night before.
\item[CnT] 	Paul was staying in a bad part of town.	He	wasn't wise	to stay at home at night.
\item[CnT] 	Paul lost his wallet with his credit cards.	He	wasn't wise	to cancel the credit cards immediately.
\item[CnF] 	Paul lost his wallet with his credit cards.	He	wasn't wise	to wait a week to cancel the credit cards.
\item[CnF] 	Paul ran a marathon on Sunday.	He	wasn't wise	to get very drunk the night before.
\item[CnF] 	Paul was staying in a bad part of town.	He	wasn't wise	to go out alone at night.
\item[CxT] 	Paul wanted to keep his excellent employee happy.	He	wasn't wise	to promote her.
\item[CxT] 	Paul bought an expensive TV.	He	wasn't wise	to ask about the return policy.
\item[CxT] 	Paul went on a hike in the Alps.	He	wasn't wise	to wear hiking boots.
\item[CxF] 	Paul had an inefficient employee.	He	wasn't wise	to promote her.
\item[CxF] 	Paul bought some heroin from a street dealer.	He	wasn't wise	to ask about the return policy.
\item[CxF] 	Paul went into the public pool.	He	wasn't wise	to wear hiking boots.

\end{enumerate}

\end{itemize}


\section{Materials of Experiment 2}\label{a-Exp2}

The following two control stimuli were used in Experiment 2: we expected participants to respond `yes' to the control stimuli because main clause content is at-issue.

\begin{exe}
\ex\label{control} Main clause control stimuli
\begin{xlist}
\ex
\begin{xlist}
\exi{Debby:} Mary brought a fruit salad.

\exi{Harry:} Are you sure?

\exi{Debby:} Yes, I am sure that Mary brought a fruit salad.
\end{xlist}
\ex 
\begin{xlist}
\exi{Debby:} Phillip, a yoga teacher, is wearing sweat pants.

\exi{Harry:} Are you sure?

\exi{Debby:} Yes, I am sure that Phillip is wearing sweat pants.
\end{xlist}
\end{xlist}
\end{exe}

The following four stimuli assessed the at-issueness of the content of a nominal appositive, of the possession content of a possessive noun phrase, of the content of the complement of the emotive predicate {\em be annoyed} and of the content of the complement of the cognitive change-of-state predicate {\em discover}. 

\begin{exe}
\ex\label{proj} Projective content control stimuli\footnote{The projective contents of these stimuli differed in the extent to which they received `no' ratings (indicating not-at-issueness): 94\% (64 of 68) for the appositive content in (\ref{proj}a), 90\% (61 of 68 for the possession content in (\ref{proj}b), 75\% (51 of 68) for the content of the complement of {\em be annoyed} in (\ref{proj}c) and 44\% (30 of 68) for the content of the  complement of {\em discover} in (\ref{proj}d). We can compare these findings with the findings reported for similar contents in \citealt{tbd-variability}, whose Exp.~2a used the same diagnostic for at-issueness, but participants gave ratings on a slider. Further differences between the two experiments is that \citealt{tbd-variability} included more than one item for each of the projective contents but only included EASs with {\em stupid}. With these caveats in mind, we can note a similarity in the findings of the two experiments: as in the experiment reported on here, \citet{tbd-variability} found that the possession content and appositive content were most not-at-issue (mean ratings of .83 and .82, respectively), followed by the content of the  complement of {\em be annoyed} (mean: .78) and the prejacent of EASs (mean: .7), and finally the content of the  complement of {\em discover} (mean .47). This finding suggests that the prejacent of EASs overall is not highly not-at-issue, which converges with the finding of the web-based corpus study (Appendix \ref{a-corpus}) that the prejacent of NEAS is not highly projective.}

\begin{xlist}

\ex Nominal appositive
\begin{xlist}
\exi{Debby:} Sue, a teacher, runs three times a week.

\exi{Harry:} Are you sure?

\exi{Debby:} Yes, I am sure that Sue is a teacher.
\end{xlist}


\ex Possessive noun phrase
\begin{xlist}
\exi{Debby:} Larry is flirting with your neighbor.

\exi{Harry:} Are you sure?

\exi{Debby:} Yes, I am sure that you have a neighbor.
\end{xlist}

\ex Emotive predicate {\em be annoyed} 
\begin{xlist}
\exi{Debby:} Tamara is annoyed that the pizza is gone.

\exi{Harry:} Are you sure?

\exi{Debby:} Yes, I am sure that the pizza is gone.
\end{xlist}

\ex Cognitive change-of-state predicate {\em discover}
\begin{xlist}
\exi{Debby:} Paula discovered that her husband is cheating.

\exi{Harry:} Are you sure?

\exi{Debby:} Yes, I am sure that Paula's husband is cheating.
\end{xlist}
\end{xlist}
\end{exe}

There were 6 target stimuli for each of the 10 evaluative adjectives: for 3 stimuli, the generalization was neutral (N); for 3, the generalization was taken to follow from the common ground (F). For each 3-turn stimulus, we provide the first and third turn, leaving out the second turn ({\em Are you sure?}).

\begin{itemize}[leftmargin=10pt,itemsep=-1pt]

\item {\bf brave}

\begin{enumerate}[leftmargin=10pt,topsep=0pt,itemsep=0pt]

\item[N]  	Greg was brave to take the job.	Yes, I am sure that he took the job.
\item[N]  	Greg was brave to watch them.	Yes, I am sure that he watched them.
\item[N]  	Greg was brave to agree to do it.	Yes, I am sure that he agreed to do it.
\item[F]  	Greg was brave to publish a controversial article under his own name.	Yes, I am sure that he published the article under his own name.
\item[F]  	Greg was brave to save a child from drowning in the river.	Yes, I am sure that he saved the child.
\item[F]  	Greg was brave to take on the five men attacking him.	Yes, I am sure that he took on the five men.

\end{enumerate}

\item {\bf foolish}

\begin{enumerate}[leftmargin=10pt,topsep=0pt,itemsep=0pt]

\item[N]  	Kate was foolish to wear that dress.	Yes, I am sure that she wore that dress.
\item[N]  	Kate was foolish to bring a snack.	Yes, I am sure that she brought a snack.
\item[N]  	Kate was foolish to go shopping.	Yes, I am sure that she went shopping.
\item[F]  	Kate was foolish to ignore her doctor's recommendations.	Yes, I am sure that she ignored his recommendations.
\item[F]  	Kate was foolish to wear high heels at the playground.	Yes, I am sure that she wore high heels at the playground.
\item[F]  	Kate was foolish to go for a run with a broken ankle.	Yes, I am sure that she went for a run with a broken ankle.

\end{enumerate}

\item {\bf fortunate}

\begin{enumerate}[leftmargin=10pt,topsep=0pt,itemsep=0pt]

\item[N]  	Sue was fortunate to meet this man.	Yes, I am sure that she met him.
\item[N]  	Sue was fortunate to be wearing flip-flops.	Yes, I am sure that she was wearing flip-flops.
\item[N]  	Sue was fortunate to attend the workshop.	Yes, I am sure that she attended the workshop.
\item[F]  	Sue was fortunate to speak French when she went to France.	Yes, I am sure that she spoke French when she went to France.
\item[F]  	Sue was fortunate to have a great singing voice.	Yes, I am sure that she had a great singing voice.
\item[F]  	Sue was fortunate to get upgraded to first class on her flight to Asia.	Yes, I am sure that she was upgraded to first class.

\end{enumerate}

\item {\bf lucky}

\begin{enumerate}[leftmargin=10pt,topsep=0pt,itemsep=0pt]

\item[N]  	Eve was lucky to grow up in that city.	Yes, I am sure that she grew up in that city.
\item[N]  	Eve was lucky to wake up.	Yes, I am sure that she woke up.
\item[N]  	Eve was lucky to get into that college.	Yes, I am sure that she got into that college.
\item[F]  	Eve was lucky to get the lead in a major TV show.	Yes, I am sure that she got the lead.
\item[F]  	Eve was lucky to win the lottery.	Yes, I am sure that she won the lottery.
\item[F]  	Eve was lucky to inherit a fortune from her uncle. 	Yes, I am sure that she inherited a fortune from her uncle.

\end{enumerate}

\item {\bf mean}

\begin{enumerate}[leftmargin=10pt,topsep=0pt,itemsep=0pt]

\item[N]  	Jack was mean to wake up his wife.	Yes, I am sure that he woke her up.
\item[N]  	Jack was mean to laugh.	Yes, I am sure that he laughed.
\item[N]  	Jack was mean to turn off the movie.	Yes, I am sure that he turned off the movie.
\item[F]  	Jack was mean to laugh when he bumped his shopping cart into a woman.	Yes, I am sure that he laughed when he bumped his cart into her.
\item[F]  	Jack was mean to turn off the movie his wife was watching.	Yes, I am sure that he turned off the movie she was watching.
\item[F]  	Jack was mean to give a milkshake to his lactose intolerant daughter.	Yes, I am sure that he gave her a milkshake.

\end{enumerate}

\item {\bf polite}

\begin{enumerate}[leftmargin=10pt,topsep=0pt,itemsep=0pt]

\item[N]  	Chad was polite to wait for her.	Yes, I am sure that he waited for her.
\item[N]  	Chad was polite to close the door.	Yes, I am sure that he closed the door.
\item[N]  	Chad was polite to laugh.	Yes, I am sure that he laughed.
\item[F]  	Chad was polite to applaud the street comedian.	Yes, I am sure that he applauded the comedian.
\item[F]  	Chad was polite to bring his grandmother a gift for her birthday.	Yes, I am sure that he brought her a gift.
\item[F]  	Chad was polite to apologize to his insulted wife.	Yes, I am sure that he apologized to her.

\end{enumerate}

\item {\bf rude}

\begin{enumerate}[leftmargin=10pt,topsep=0pt,itemsep=0pt]

\item[N]  	Ann was rude to say that.	Yes, I am sure that she said that.
\item[N]  	Ann was rude to ask that.	Yes, I am sure that she asked that.
\item[N]  	Ann was rude to change the song.	Yes, I am sure that she changed the song.
\item[F]  	Ann was rude to eat the pasta with her fingers.	Yes, I am sure that she ate the pasta with her fingers.
\item[F]  	Ann was rude to laugh at her husband's pain.	Yes, I am sure that she laughed at his pain.
\item[F]  	Ann was rude to ignore her friendly new neighbor.	Yes, I am sure that she ignored him.

\end{enumerate}

\item {\bf smart}

\begin{enumerate}[leftmargin=10pt,topsep=0pt,itemsep=0pt]

\item[N]  	Jane was smart to stay home.	Yes, I am sure that she stayed home.
\item[N]  	Jane was smart to say no.	Yes, I am sure that she said no.
\item[N]  	Jane was smart to go to the store.	Yes, I am sure that she went to the store.
\item[F]  	Jane was smart to stay home when she had the flu.	Yes, I am sure that she stayed home when she had the flu.
\item[F]  	Jane was smart to change her passwords when her computer was hacked.	Yes, I am sure that she changed her passwords when her computer was hacked.
\item[F]  	Jane was smart to report her stalker.	Yes, I am sure that she reported him.

\end{enumerate}

\item {\bf stupid}

\begin{enumerate}[leftmargin=10pt,topsep=0pt,itemsep=0pt]

\item[N]  	Zack was stupid to go to the store.	Yes, I am sure that he went to the store.
\item[N]  	Zack was stupid to leave the bar.	Yes, I am sure that he left the bar.
\item[N]  	Zack was stupid to dance like that.	Yes, I am sure that he danced like that.
\item[F]  	Zack was stupid to go to school with the measles.	Yes, I am sure that she went to school with the measles.
\item[F]  	Zack was stupid to drive home completely drunk.	Yes, I am sure that he drove home completely drunk.
\item[F]  	Zack was stupid to use the hair dryer in the bath tub.	Yes, I am sure that he used the hair dryer in the bath tub.

\end{enumerate}

\item {\bf wise}

\begin{enumerate}[leftmargin=10pt,topsep=0pt,itemsep=0pt]

\item[N]  	Paul was wise to reprimand his employee.	Yes, I am sure that he reprimanded her.
\item[N]  	Paul was wise to wear hiking boots.	Yes, I am sure that he wore his hiking boots.
\item[N]  	Paul was wise to stay at home.	Yes, I am sure that he stayed at home.
\item[F]  	Paul was wise to promote his excellent employee.	Yes, I am sure that he promoted his excellent employee.
\item[F]  	Paul was wise to go to bed early the night before the marathon.	Yes, I am sure that he went to bed early that night.
\item[F]  	Paul was wise to wear hiking boots on his hike in the Alps.	Yes, I am sure that he wore hiking boots on his hike.

\end{enumerate}

\end{itemize}

\bibliographystyle{cslipubs-natbib}
\bibliography{bibliography}

\end{document}
